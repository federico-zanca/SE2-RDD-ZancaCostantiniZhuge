CodeKataBattle (CKB) emerges as a pivotal platform in the evolving landscape of software development education. 
As the paradigm shifts towards enhancing coding proficiency, CKB provides a dynamic arena where students engage in collaborative code kata battles, 
refining their skills through hands-on challenges. 

Educators orchestrate these battles within tournaments, defining parameters such as group sizes, deadlines, and scoring configurations. 
Facilitating a test-first approach, CKB seamlessly integrates with GitHub, automating workflows and enabling real-time evaluations. 
As scores evolve, students and educators gain insights into individual and team performances, fostering a competitive yet collaborative learning environment. 

Beyond individual battles, CKB aggregates personal tournament scores, offering a comprehensive view of students' progress. 
Introducing gamification with badges, educators recognize achievements, enhancing the overall learning experience. 
In essence, CodeKataBattle propels software development education into a new era, blending competition, collaboration, and skill refinement within a cutting-edge platform.
\newpage


\section{Purpose}
\label{sec:purpose}%

\subsection{Goals}
\label{subsec:goals}%
\newcounter{g}
\setcounter{g}{1}
\newcommand{\cg}{\theg\stepcounter{g}}
\verb|CodeKataBattle(CKB)| platform serves as a dynamic and interactive space for students and educators, 
promoting continuous learning, healthy competition, and recognition of individual and team accomplishments in the realm of software development.

Students can engage in coding challenges by joining battles individually or forming teams within specified limits.
By actively contributing to code repositories on GitHub, following a test-first approach, they strive to achieve high scores in battles, contributing to their personal tournament ranking.
Eventually work towards earning gamification badges based on their performance and achievements in tournaments.

Educators instead, develop and publish code kata battles, setting criteria for evaluation and deadlines.
They will assess student submissions, providing manual scores if required and shaping the final rankings, also defining gamification badges with specific rules and criteria to recognize outstanding achievements.
Further more they will oversee the progression of tournaments, including opening, closing, and notifying participants.

Below there's a table that lists all the goals of the \verb|CKB| platform:
\begin{center}
    \begin{longtable}{ |l|p{0.9\linewidth}| }
        \hline
        \textbf{ID} & \textbf{Description}                                                                   \\
        \hline
        G\cg        & The platform should allow educators to set up tournaments.\\
        \hline
        G\cg        & The platform should allow educators to set up code kata battles with configurable parameters.\\
        \hline
        G\cg        & The platform should allow third party platforms integration.\\
        \hline
        G\cg        & The platform should allow students to join battles individually or form teams within specified size limits.\\
        \hline
        G\cg        & The platform should allow users to see real time updates on current tournaments and battles.\\
        \hline
        G\cg        & The platform should have automated evaluations of code submissions.\\
        \hline
        G\cg        & The platform should allow educators to manually evaluate and assign scores for optional factors at the end of each battle.                     \\
        \hline
        G\cg        & The platform should allow users to visualize informations about another user.\\
        \hline
        G\cg        & The platform should allow educators to create new gamification badges.\\
        \hline
        \caption{The goals.}
        \label{tab:goals_tab}%
    \end{longtable}
\end{center}

%\newpage


\section{Scope}
\label{sec:scope}%
The scope of the requirements analysis and specification document - RASD - is to provide a detailed description of the requirements for the project.
It outlines what the system or product should do and how it should behave, as well as any constraints or limitations on its design and implementation.
The RASD is based on a thorough analysis of the needs and requirements of the stakeholders.
The scope of the RASD will cover all of the essential aspects of the system and provide a clear and detailed roadmap for its development.

\subsection{World phenomena}
\label{subsec:world_phenomena}%
\newcounter{wp}
\setcounter{wp}{1}
\newcommand{\cwp}{\thewp\stepcounter{wp}}
\begin{center}
    \begin{longtable}{ |l|p{0.8\linewidth}| }
        \hline
        \textbf{ID} & \textbf{Description}                                                \\
        \hline
        WP\cwp      & Educators create new tournaments for coding challenges.                          \\
        \hline
        WP\cwp      & Students seek opportunities to improve their software development skills.   \\
        \hline
        WP\cwp      & Educators define code kata battles with project details and evaluation criteria. \\
        \hline
        WP\cwp      & Students aim to participate in code kata battles individually or form teams.          \\
        \hline
        WP\cwp      & Students are informed of upcoming battles when subscribed to a tournament.                \\
        \hline
        WP\cwp      & Students form teams within specified limits for a code kata battle.                 \\
        \hline
        WP\cwp      & Students aim to visualize their personal tournament scores and earned badges.         \\
        \hline
        WP\cwp      & Educators define gamification badges and rules when creating a tournament.      \\
        \hline
        WP\cwp      & At the end of each battle, the CKB platform notifies students of the final battle rank and updates personal tournament scores.                       \\
        \hline
        WP\cwp      & Educators can close tournaments, and the platform notifies all students when the final tournament rank becomes available. \\
        \hline
        WP\cwp      & Students and educators seek information about ongoing tournaments, ranks, and profiles with badges.                      \\
        \hline
        \caption{World Phenomenas.}
        \label{tab:worldph_tab}%
    \end{longtable}
\end{center}

\subsection{Shared phenomena}
\label{subsec:shared_phenomena}%
\newcounter{sp}
\setcounter{sp}{1}
\newcommand{\csp} {\thesp\stepcounter{sp}}
\begin{center}
    \begin{longtable}{ |l|p{0.5\linewidth}|l|l| }
        \hline
        \textbf{ID} & \textbf{Description}                                                                                                          & \textbf{Controller} & \textbf{Observer} \\
        \hline
        SP\csp      & Student subscribes to the CKB platform for notifications.\                                                                    & Student                 & CKB Platform             \\
        \hline
        SP\csp      & Educator creates a new tournament, specifying details and criteria.\                                                          & Educator                 & CKB Platform            \\
        \hline
        SP\csp      & CKB Platform notifies subscribed students about a new tournament.\                                                            & CKB Platform                & Student             \\
        \hline
        SP\csp      & Student forms a team within specified limits for a code kata battle.\                                                         & Student               & CKB Platform               \\
        \hline
        SP\csp      & Educator sets up a code kata battle, including project details and evaluation criteria.\                                      & Educator                 & CKB Platform            \\
        \hline
        SP\csp      & CKB Platform generates a GitHub repository for a code kata battle.                                                            & CKB Platform               & Student               \\
        \hline
        SP\csp      & Student forks the GitHub repository and sets up automated workflows.                                                          & Student                 & CKB Platform             \\
        \hline
        SP\csp      & Student pushes commits to GitHub, triggering automated evaluation.                                                            & Student              & CKB Platform               \\
        \hline
        SP\csp      & Automated evaluation includes functional correctness, timeliness, and code quality.                                           & CKB Platform               & Student               \\
        \hline
        SP\csp      & Educator manually evaluates optional factors, such as personal scores.                                                        & Educator               & CKB Platform              \\
        \hline
        SP\csp      & CKB Platform updates battle scores in real-time based on new commits.                                                         & CKB Platform              & Student               \\
        \hline
        SP\csp      & Students and educators observe the evolving rank during a battle.                                                             & Student               & CKB Platform             \\
        \hline
        SP\csp      & CKB Platform notifies students when the final battle rank is available.                                                       & CKB Platform              & Student               \\
        \hline
        SP\csp      & CKB Platform updates personal tournament scores at the end of a battle.                                                       & CKB Platform               & Student              \\
        \hline
        SP\csp      & Educator closes a tournament, and the platform notifies students of the final rank.                                           & Educator                & CKB Platform            \\
        \hline
        SP\csp      & Educator defines gamification badges and rules when creating a tournament.                                                    & Educator              & CKB Platform              \\
        \hline
        SP\csp      & Students earn gamification badges based on their performance and achievements.                                                & Student                 & CKB Platform            \\
        \hline
        SP\csp      & Students and educators visualize ongoing tournaments, ranks, and profiles with badges.                                        & Student               & CKB Platform             \\
        \hline
        
        \caption{Shared Phenomenas.}
        \label{tab:sharedph_tab}%
    \end{longtable}
\end{center}


\section{Definition, Acronyms, Abbreviations}
\label{sec:definition_acronyms_abbreviations}%
\begin{table}[H]
    \begin{center}
        \begin{tabular}{ |l|l| }
            \hline
            \textbf{Acronyms} & \textbf{Definition}                              \\
            \hline
            CKB              & Code Kata Battle                      \\
            \hline
            RASD              & Requirements Analysis and Specification Document \\
            \hline
        \end{tabular}
        \caption{Acronyms used in the document.}
        \label{tab:acronyms}%
    \end{center}
\end{table}


\section{Revision history}
\label{sec:revision_history}%
This version of RASD differs from the previous one for:
\begin{itemize}
    \item In section 1.2, we have inserted a brief description about the scope of the document.
    \item In section 3.1.3, we have added the Navigator System software interface.
    \item In section 3.2.1, we have added the requirements R70 to the table and updated the requirement R39.
    \item In section 3.4, we have updated some sequence diagrams to make them coherent with the ones presented in the DD\@.
\end{itemize}


\section{Reference Documents}
\label{sec:reference_documents}%
%%\begin{itemize}
%%\end{itemize}


\section{Document Structure}
\label{sec:document_structure}%
The document is structured in six sections, as described below.

First section introduce the goals of the project, purposes, and a brief analysis on world and shared phenomena;
abbreviations and definitions useful to understand the problem are listed as well.

The following section, the second one, provides an overall description of the problem: here further
details on domain and scenarios are included, aside from more product and user characteristics, assumptions,
dependencies and constraints.

Later on, the third section focuses on the specific requirements and provides a more detailed analysis of external
interface requirements, functional requirements and performance requirements.

Lastly, the fourth section provides a formal analysis, using alloy.
This chapter is crucial to prove the correctness of the model described in the previous sections, and should focus on
reporting results of the checks performed and meaningful assertions.

Section five reports the effort spent by each group member in the redaction of this document, meanwhile the last
section simply lists bibliography references and other resources used to redact this document.
