\section{External Interface Requirements}
\label{sec:external_interface_requirements}%

\subsection{User Interfaces}
\label{subsec:user_interfaces}%
The \verb|CodeKataBattle (CKB)| platform is accessed via an intuitive and responsive web interface compatible with major browsers (Chrome, Firefox, Safari, etc.).
Educators enjoy a dedicated dashboard for effortless creation and management of tournaments, battles, and badges. This dashboard provides a comprehensive view of ongoing battles, tournament scores, and badge achievements.
Students utilize a user-friendly dashboard for team formation, battle participation, and progress tracking. It streamlines team formation, displays upcoming battles, current ranks, and summarizes earned badges.
GitHub seamlessly integrates into the platform for code versioning and automated testing. Students can easily fork repositories, set up GitHub Actions, and monitor build and test results within the \verb|CKB| platform.
To keep all stakeholders informed, the platform employs a robust notification system. This system supports both email notifications and in-platform alerts, 
ensuring timely updates for educators and students on critical events like upcoming deadlines or changes in battle status.

\subsection{Hardware Interfaces}
\label{subsec:hardware_interfaces}%
The \verb|CKB| platform prioritizes accessibility by ensuring compatibility across a diverse range of devices. 
Users can seamlessly access the platform from desktop computers, laptops, tablets, and smartphones. This inclusivity caters to the varied preferences and device usage patterns of our user base. 
The platform's responsive design ensures that the user interface adapts fluidly to different screen sizes, providing an optimal experience regardless of the device used. 
This commitment to device compatibility aims to enhance user convenience and flexibility, promoting a versatile and user-centric engagement with the \verb|CKB| platform.

\subsection{Software Interfaces}
\label{subsec:software_interfaces}%
The \verb|CKB| platform seamlessly communicates with GitHub via APIs, enabling functionalities such as repository creation, commit tracking, and automated test processes.
To ensure a secure integration, the platform should smoothly connect with GitHub APIs, facilitating automated workflows triggered by student commits.
Additionally, the platform harnesses static analysis tools to assess code quality comprehensively.
Incorporating these tools seamlessly, educators can tailor automated evaluations by configuring specific aspects like security, reliability, and maintainability. This flexibility ensures a nuanced understanding of the code's overall quality.

\subsection{Communication Interfaces}
\label{subsec:communication_interfaces}%
The platform actively engages in communication with students, delivering notifications, battle updates, and final results in a secure manner through HTTPS. A reliable messaging system ensures timely information dissemination to students.
Similarly, educators stay well-informed through the platform, receiving notifications and updates on battle progress and final results. Secure communication channels, similar to those used for students, guarantee the confidentiality and reliability of information relayed to educators.
For the configuration of badges and rules, educators seamlessly use the platform to define new badges, rules, and associated variables. Ensuring a user-friendly interface, educators can make real-time adjustments, with changes promptly reflecting across the platform. This intuitive process empowers educators to tailor the platform to their evolving requirements effortlessly.

\section{Functional Requirements}
\label{sec:functional_requirements}%

\subsection{Requirements}
\label{subsec: requirements}%
The \verb|CKB| platform offers several functionalities to both educators and students.
In the following table they are listed all the detected requirements that the platform should respect in order to guarantee
the satisfiability of the goals:
\newpage
\newcounter{req}
\setcounter{req}{1}
\newcommand{\creq}{\thereq\stepcounter{req}}
\textbf{Educators}
\begin{center}
    \begin{longtable}{|l|p{0.9\linewidth}|}
        \hline
        \textbf{ID} & \textbf{Description}                                                                                                                             \\
        \hline
        R\creq      & The \verb|CKB| platform shall allow educators to create an account.                                                                    \\
        \hline
        R\creq      & The \verb|CKB| platform shall allow educators to log in.                                                                                 \\
        \hline
        R\creq      & The \verb|CKB| platform shall allow educators to create a new tournament.                                                                \\
        \hline
        R\creq      & The \verb|CKB| platform shall allow educators to set the minimum and maximum number of students per group for a tournament.                                                        \\
        \hline
        R\creq      & The \verb|CKB| platform shall allow educators to upload a code kata battle.                                                                         \\
        \hline
        R\creq      & The \verb|CKB| platform shall allow educators to grant permissions to other educators to create battles within a specific tournament.                                                      \\
        \hline
        R\creq      & The \verb|CKB| platform shall enable educators to include a battle to a specific tournament.                                                      \\
        \hline
        R\creq      & The \verb|CKB| platform shall allow educators to include a description for a battle.                                                                \\
        \hline
        R\creq      & The \verb|CKB| platform shall allow educators to include a software project with test cases.                                                                \\
        \hline
        R\creq      & The \verb|CKB| platform shall allow educators to set a registration deadline for a battle within a tournament.                                          \\
        \hline
        R\creq      & The \verb|CKB| platform shall allow educators to set a final submission deadline for a battle within a tournament.                                                                  \\
        \hline
        R\creq      & The \verb|CKB| platform shall allow educators to set additional configurations for scoring, including functional aspects and quality level criteria.                               \\
        \hline
        R\creq      & The \verb|CKB| platform shall allow educators to close a tournament.                                                  \\
        \hline
        R\creq      & The \verb|CKB| platform shall allow an educator to define optional manual evaluation criteria for score assignment in battles.   \\
        \hline
        R\creq      & The \verb|CKB| platform shall allow educators to manually evaluate and assign scores to teams.                                           \\
        \hline
        R\creq      & The \verb|CKB| platform shall allow educators and students to visualize the gamification badges.                                                      \\
        \hline
        R\creq      & The \verb|CKB| platform shall allow educators to define new badges for gamification.                                                             \\
        \hline
        R\creq      & The \verb|CKB| platform shall allow educators to define new rules associated with the badges.                                                      \\
        \hline
        R\creq      & The \verb|CKB| platform shall allow educators to define new variables associated with the badges.                                                      \\
        \hline
        R\creq      & The \verb|CKB| platform shall allow all students and educators to see the ranking of each ongoing tournament with the score of each student subscribed.\\
        \hline
        \caption{Educator Requirements.}
        \label{tab: req}%
    \end{longtable}
\end{center}

\textbf{Students}
\begin{center}
    \begin{longtable}{|l|p{0.9\linewidth}|}
        \hline
        \textbf{ID} & \textbf{Description}                                                                                                                             \\
        \hline
        R\creq      & The \verb|CKB| platform shall not allow students to participate a tournament after the registration deadline.                                                      \\
        \hline
        R\creq      & The \verb|CKB| platform shall not allow students to participate a battle after the registration deadline.                                                      \\
        \hline
        R\creq      & The \verb|CKB| platform shall allow students to subscribe to a tournament within a specified deadline.                                                  \\
        \hline
        R\creq      & The \verb|CKB| platform shall allow students to create teams for a specific battle within a tournament.                                                      \\
        \hline
        R\creq      & The \verb|CKB| platform shall allow students to join teams for a specific battle within a tournament.                                                      \\
        \hline
        R\creq      & The \verb|CKB| platform shall allow students to invite other participants to the same group.                               \\
        \hline
        R\creq      & The \verb|CKB| platform shall allow students to accept an invitation.                               \\
        \hline
        R\creq      & The \verb|CKB| platform shall allow students to reject an invitation.                               \\
        \hline
        R\creq      & The \verb|CKB| platform shall allow students to join a battle without a team.                               \\
        \hline
        \caption{Student Requirements.}
        \label{tab: req}%
    \end{longtable}
\end{center}

\textbf{Platform}
\begin{center}
    \begin{longtable}{|l|p{0.9\linewidth}|}
        \hline
        \textbf{ID} & \textbf{Description}                                                                                                                             \\
        \hline
        R\creq      & The \verb|CKB| platform shall notify all subscribed students of a new battle and its details within a specific tournament.                               \\
        \hline
        R\creq      & The \verb|CKB| platform shall notify all subscribed students of a new tournament and its details.                               \\
        \hline
        R\creq      & The \verb|CKB| platform shall create a GitHub repository for each battle.                                        \\
        \hline
        R\creq      & send a link to the Github repository associated to a battle to all members of subscribed teams upon expiration of the registration deadline. \\
        \hline
        R\creq      & The \verb|CKB| platform shall be able to be informed of new students' commits by Github Actions workflows. \\
        \hline
        R\creq      & The \verb|CKB| platform shall be able to pull the latest sources from the forks of the Github repository provided. \\
        \hline
        R\creq      & The \verb|CKB| platform shall be able to run the testcases on the code uploaded by students and determine if the code is a valid solution for the exercise.\\
        \hline
        R\creq      & The \verb|CKB| platform shall inform students of the mandatory automated evaluation criteria, including functional aspects, timeliness, and source code quality.                                        \\
        \hline
        R\creq      & The \verb|CKB| platform shall automatically update the battle score of a team based on GitHub commits and test results.                                   \\
        \hline
        R\creq      & The \verb|CKB| platform shall automatically close a finished battle.                                                      \\
        \hline
        R\creq      & The \verb|CKB| platform shall assign or update battle scores to each team of the battle.                                                      \\
        \hline
        R\creq      & The \verb|CKB| platform shall calculate and update the personal tournament score of each student based on their performance in battles.                    \\
        \hline
        R\creq      & The \verb|CKB| platform shall be able to create a ranking of teams for every tournament.                                                      \\
        \hline
        R\creq      & The \verb|CKB| platform shall keep track of time elapsed from the start of a CKB and the final submissions of each team. \\
        \hline
        R\creq      & The \verb|CKB| platform shall be able to use static analysis tools to evaluate the quality of the code submitted by teams in terms of security, reliability, maintainability and other aspects defined by the educator who created the battle. \\
        \hline
        R\creq      & The \verb|CKB| platform shall notify all students involved in a tournament when it is closed and the final ranking is available.                                                       \\
        \hline
        R\creq      & The \verb|CKB| platform shall visualize ongoing tournaments and their ranks for all users.                                                                 \\
        \hline
        R\creq      & The \verb|CKB| platform shall display collected badges on the profile of both students and educators.                                                                    \\
        \hline
        \caption{Platform Requirements.}
        \label{tab: req}%
    \end{longtable}
\end{center}

\subsection{Mapping on goals}
\label{subsec: map_on_g}%
In the following section it is shown how the relation $R\land D \models G$ holds.
In particular, at first it is shown a traceability matrix that associates domain assumptions and requirements to each goal.
After that, to facilitate reading, the section reports the text of all the assumptions and all the requirements related to each goal.
\newcounter{mg}
\setcounter{mg}{1}
\newcommand{\cmg}{\themg\stepcounter{mg}}
\begin{center}
    \begin{longtable}{|p{0.06\linewidth}|p{0.34\linewidth}|p{0.6\linewidth}|}
        \hline
        \textbf{Goal} & \textbf{Domain assumptions}                       & \textbf{Requirements}                                                               \\
        \hline
        G\cmg         & D2,D4,D6,D10                & R1,R2,R3,R4,R10,R13\\
        \hline
        G\cmg         & D1,D2                       & R1,R2,R5,R6,R7,R8,R11\\
        \hline
        G\cmg         & D2,D5                       & R32,R33,R34,R35,R38\\
        \hline
        G\cmg         & D2                          & R24,R25,R26,R27,R28,R29\\
        \hline
        G\cmg         & D2                          & R30,R31,R45,R46\\
        \hline
        G\cmg         & D1,D2,D7                    & R9,R34,R35,R36,R37,R38,R41,R44\\
        \hline
        G\cmg         & D1,D2,D3                    & R14,R15\\
        \hline
        G\cmg         & D2                          & R16,R20,R46,R47\\
        \hline
        G\cmg         & D2,D8                       & R17,R18,R19\\
        \hline
        \caption{Mapping on goals.}
        \label{tab: map_on_g}%
    \end{longtable}
\end{center}

In this section, it will be shown the functional requirements and the domain assumption related to each goal.
\begin{itemize}

    \item \textbf{{[G.1]} The platform should allow educators to set up tournaments.}
    \begin{itemize}
        \item \textbf{[R.1]} The \verb|CKB| platform shall allow educators to create an account.
        \item \textbf{[R.2]} The \verb|CKB| platform shall allow educators to log in.
        \item \textbf{[R.3]} The \verb|CKB| platform shall allow educators to create a new tournament. 
        \item \textbf{[R.4]} The \verb|CKB| platform shall allow educators to set the minimum and maximum number of students per group for a tournament. 
        \item \textbf{[R.10]} The \verb|CKB| platform shall allow educators to set a registration deadline for a battle within a tournament.
        \item \textbf{[R.13]} The \verb|CKB| platform shall allow educators to close a tournament.
        \item \textbf{[D.2]} All users of the CKB platform, both educators and students, are assumed to have access to a stable internet connection. This is
        necessary fir accessing the platform, downloading code kata, submitting code, and receiving notification.
        \item \textbf{[D.4]}  Educators managing a torunament will not lose access to their institutional email during the whole duration of the tournament itself.
        \item \textbf{[D.6]} Educators are capable of creating and closing tournaments, without leaving them open undefinetly when no more battles are scheduled or in session.
        \item \textbf{[D.10]} Educators and students are linked to one and only one institutional email.
    \end{itemize}

    \item \textbf{{[G.2]} The platform should allow educators to set up code kata battles with configurable parameters. }
        \begin{itemize}
            \item \textbf{[R.1]} The \verb|CKB| platform shall allow educators to create an account.
            \item \textbf{[R.2]} The \verb|CKB| platform shall allow educators to log in.
            \item \textbf{[R.5]} The \verb|CKB| platform shall allow educators to upload a code kata battle. 
            \item \textbf{[R.6]} The \verb|CKB| platform shall allow educators to grant permissions to other educators to create battles within a specific tournament.
            \item \textbf{[R.7]} The \verb|CKB| platform shall enable educators to include a battle to a specific tournament.
            \item \textbf{[R.8]} The \verb|CKB| platform shall allow educators to include a description for a battle.     
            \item \textbf{[R.11]} The \verb|CKB| platform shall allow educators to set a final submission deadline for a battle within a tournament.
            \item \textbf{[D.1]} Educators using CKB platform have the necessary technical knowledge and skills to create and manage code kata battles. 
            This includes the ability to create programming exercises, write test cases, and set up build automation scripts.
            \item \textbf{[D.2]} All users of the CKB platform, both educators and students, are assumed to have access to a stable internet connection. This is
            necessary fir accessing the platform, downloading code kata, submitting code, and receiving notification.
        \end{itemize}

        \item \textbf{{[G.3]} The platform should allow third party platforms integration. }
        \begin{itemize}
            \item \textbf{[R.32]} The \verb|CKB| platform shall create a GitHub repository for each battle.
            \item \textbf{[R.33]} send a link to the Github repository associated to a battle to all members of subscribed teams upon expiration of the registration deadline. 
            \item \textbf{[R.34]} The \verb|CKB| platform shall be able to be informed of new students' commits by Github Actions workflows.
            \item \textbf{[R.35]} The \verb|CKB| platform shall be able to pull the latest sources from the forks of the Github repository provided.
            \item \textbf{[R.38]} The \verb|CKB| platform shall automatically update the battle score of a team based on GitHub commits and test results.

            \item \textbf{[D.1]} Educators using \verb|CKB| platform have the necessary technical knowledge and skills to create and manage code kata battles. 
            This includes the ability to create programming exercises, write test cases, and set up build automation scripts.
            \item \textbf{[D.2]} All users of the CKB platform, both educators and students, are assumed to have access to a stable internet connection. This is
            necessary fir accessing the platform, downloading code kata, submitting code, and receiving notification.  
            \item \textbf{[D.5]} Educators and students have a good understanding of GitHub and GitHub actions.

        \end{itemize}

        \item \textbf{{[G.4]} The platform should allow students to join battles individually or form teams within specified size limits. }
        \begin{itemize}
            \item \textbf{[R.24]} The \verb|CKB| platform shall allow students to create teams for a specific battle within a tournament. 
            \item \textbf{[R.25]} The \verb|CKB| platform shall allow students to join teams for a specific battle within a tournament.
            \item \textbf{[R.26]} The \verb|CKB| platform shall allow students to invite other participants to the same group.
            \item \textbf{[R.27]} The \verb|CKB| platform shall allow students to accept an invitation.
            \item \textbf{[R.28]} The \verb|CKB| platform shall allow students to reject an invitation. 
            \item \textbf{[R.29]} The \verb|CKB| platform shall allow students to join a battle without a team.
            \item \textbf{[D.2]} All users of the CKB platform, both educators and students, are assumed to have access to a stable internet connection. This is
            necessary fir accessing the platform, downloading code kata, submitting code, and receiving notification.
        \end{itemize}

        \item \textbf{{[G.5]} The platform should allow users to see real time updates on current tournaments and battles.}
        \begin{itemize}
            \item \textbf{[R.30]} The \verb|CKB| platform shall notify all subscribed students of a new battle and its details within a specific tournament.
            \item \textbf{[R.31]} The \verb|CKB| platform shall notify all subscribed students of a new tournament and its details.
            \item \textbf{[R.45]} The \verb|CKB| platform shall notify all students involved in a tournament when it is closed and the final ranking is available. 
            \item \textbf{[R.46]} The \verb|CKB| platform shall visualize ongoing tournaments and their ranks for all users.
            \item \textbf{[D.2]} All users of the CKB platform, both educators and students, are assumed to have access to a stable internet connection. This is
            necessary fir accessing the platform, downloading code kata, submitting code, and receiving notification.
        \end{itemize}

        \item \textbf{{[G.6]} The platform should have automated evaluations of code submissions. }
        \begin{itemize}
            \item \textbf{[R.9]} The \verb|CKB| platform shall allow educators to include a software project with test cases.
            \item \textbf{[R.34]} The \verb|CKB| platform shall be able to be informed of new students' commits by Github Actions workflows.
            \item \textbf{[R.35]} The \verb|CKB| platform shall be able to pull the latest sources from the forks of the Github repository provided.
            \item \textbf{[R.36]} The \verb|CKB| platform shall be able to run the testcases on the code uploaded by students and determine if the code is a valid solution for the exercise.
            \item \textbf{[R.37]} The \verb|CKB| platform shall inform students of the mandatory automated evaluation criteria, including functional aspects, timeliness, and source code quality.
            \item \textbf{[R.38]} The \verb|CKB| platform shall automatically update the battle score of a team based on GitHub commits and test results.
            \item \textbf{[R.41]} The \verb|CKB| platform shall calculate and update the personal tournament score of each student based on their performance in battles.
            \item \textbf{[R.44]} The \verb|CKB| platform shall be able to use static analysis tools to evaluate the quality of the code submitted by teams in terms of security, reliability, maintainability and other aspects defined by the educator who created the battle. 
            \item \textbf{[D.1]} Educators using \verb|CKB| platform have the necessary technical knowledge and skills to create and manage code kata battles. 
            This includes the ability to create programming exercises, write test cases, and set up build automation scripts.
            \item \textbf{[D.2]} All users of the CKB platform, both educators and students, are assumed to have access to a stable internet connection. This is
            necessary fir accessing the platform, downloading code kata, submitting code, and receiving notification. 
            \item \textbf{[D.7]} The scoring system is transparent, consistent, and coherent with the criteria of the rules (quality of sources, timeliness, ...)

        \end{itemize}

        \item \textbf{{[G.7]} The platform should allow educators to manually evaluate and assign scores for optional factors at the end of each battle. }
        \begin{itemize}
            \item \textbf{[R.14]} The \verb|CKB| platform shall allow an educator to define optional manual evaluation criteria for score assignment in battles.   
            \item \textbf{[R.15]} The \verb|CKB| platform shall allow educators to manually evaluate and assign scores to teams.
            \item \textbf{[D.1]} Educators using \verb|CKB| platform have the necessary technical knowledge and skills to create and manage code kata battles. 
            This includes the ability to create programming exercises, write test cases, and set up build automation scripts.
            \item \textbf{[D.2]} All users of the CKB platform, both educators and students, are assumed to have access to a stable internet connection. This is
            necessary fir accessing the platform, downloading code kata, submitting code, and receiving notification.
            \item \textbf{[D.3]} Educators are expected to have the ability to evaluate the work done by students and assign scores if manual evaluation is required.
        \end{itemize}

        \item \textbf{{[G.8]} The platform should allow users to visualize informations about another user. }
        \begin{itemize}
            \item \textbf{[R.16]} The \verb|CKB| platform shall allow educators and students to visualize the gamification badges.   
            \item \textbf{[R.20]} The \verb|CKB| platform shall allow all students and educators to see the ranking of each ongoing tournament with the score of each student subscribed.   
            \item \textbf{[R.46]} The \verb|CKB| platform shall visualize ongoing tournaments and their ranks for all users.
            \item \textbf{[R.47]} The \verb|CKB| platform shall display collected badges on the profile of both students and educators.    
            \item \textbf{[D.2]} All users of the CKB platform, both educators and students, are assumed to have access to a stable internet connection. This is
            necessary fir accessing the platform, downloading code kata, submitting code, and receiving notification.        \end{itemize}

        \item \textbf{{[G.9]} The platform should allow educators to create new gamification badges.}
        \begin{itemize}
            \item \textbf{[R.17]} The \verb|CKB| platform shall allow educators to define new badges for gamification. 
            \item \textbf{[R.18]} The \verb|CKB| platform shall allow educators to define new rules associated with the badges.
            \item \textbf{[R.19]} The \verb|CKB| platform shall allow educators to define new variables associated with the badges.
            \item \textbf{[D.2]} All users of the CKB platform, both educators and students, are assumed to have access to a stable internet connection. This is
            necessary fir accessing the platform, downloading code kata, submitting code, and receiving notification.
            \item \textbf{[D.8]} Badge rules and varibales are well-defined and understood by the platform. They are meaningfull with respect to the tournament associated with.
        \end{itemize}
    \end{itemize}
\section{Use Case Diagrams}
\label{subsec:use_cases}
\subsection{Guest}
\label{subsec: use_case_diagrams}%
% Use Case Diagrams
\begin{figure}[H]
    \begin{center}
        \includegraphics[width=0.6\linewidth]{Images/UCD_Registration.png}
        \caption{Guest Diagram}
        \label{fig:class_diagram}%
    \end{center}
\end{figure}

\subsection{Student}
\begin{figure}[H]
    \begin{center}
        \includegraphics[width=0.8\linewidth]{Images/UCD_Student.png}
        \caption{Student Diagram}
        \label{fig:class_diagram}%
    \end{center}
\end{figure}
\subsection{Educator}
\begin{figure}[H]
    \begin{center}
        \includegraphics[width=0.8\linewidth]{Images/UCD_Educators.png}
        \caption{Educator Diagram}
        \label{fig:class_diagram}%
    \end{center}
\end{figure}

\section{Use Cases}
\label{subsec: use_case}%
\newcounter{uc}
\setcounter{uc}{1}
\newcommand{\cuc}{\theuc\stepcounter{uc}}
In this section, the primary identified use cases are elucidated. 
Each use case is accompanied by a table delineating entry conditions, event flow, exit conditions, and exceptions. 
Additionally, a sequence diagram is provided to illustrate the interactions between entities and the functions invoked. 
This comprehensive representation aims to capture the essential aspects of each use case, ensuring a clear understanding of the system dynamics within the context of the Code Kata Battle (CKB) platform.
\subsubsection*{UC\cuc . Student Registration}
\begin{center}
    \begin{longtable}{lp{0.75\linewidth}}
        \hline
        Actor            & Unregistered Student                                                                                                                                                                               \\
        \hline
        Entry conditions & The student isn't registered on the \verb|CKB| platform and clicks the sign-up button                                                                                                               \\
        \hline
        Event Flow       
        & 1. The \verb|CKB| platform prompts the unregistered student to input personal information (name, surname, email linked to an institutional profile).\\
        & 2. The unregistered student fills the form with personal information\\
        & 3. The student agrees to the platform's "Terms \& Conditions," and "Privacy Policy."\\
        & 4. The \verb|CKB| platform validates the provided student information.\\
        & 5. The \verb|CKB| platform requests the student to input a username for their profile.\\
        & 6. The student inputs a username.\\ 
        & 7. The \verb|CKB| platform validates the username.\\
        & 8. The \verb|CKB| platform sends a verification email containing a 6-digit code to the student's provided institutional email address.\\
        & 9. The \verb|CKB| platform prompts the student to input the verification code.\\
        & 10. The student inputs the verification code.\\
        & 11. The \verb|CKB| platform communicates the outcome of the student's registration.\\
        \hline
        Exit condition   & An account is created.   \\                                                                                                                                                                           
        \hline
        Exceptions   
        & 3.1. The student does not agree to the platform's "Terms \& Conditions," and "Privacy Policy."\\
            & The \verb|CKB| platform shows a message asking the user to agree to such "Terms \& Conditions" stopping the sign up operation.  \\                                                                                                   
        & 4.1. The \verb|CKB| platform is unable to validate the student's personal information.  \\                                                                                                         
            & In these cases, the unregistered student receives a notification with an error message.   \\
        & 7.1. The student's provided username is already in use.           \\                                                                                                
            & The \verb|CKB| platform shows a message asking the user to choose another username.   \\
        & 7.2. The student's provided username is in a non acceptable format    \\
            & The \verb|CKB| platform shows a message asking the user to choos another username    \\
      % email not received by user                                                                                        \\
        & 10.1. The student inputs an incorrect verification code.      \\                                                           
            & The \verb|CKB| platform shows a message asking the user to input the correct verification code.  \\
                                                                                        
        \hline
        \caption{Student Registration Use Case.}
        \label{tab: student_registration_use_case}
    \end{longtable}

    %QUI SEQUENCE DIAGRAM
   
        \begin{figure} [H]
            \begin{center}
                \includegraphics[width=0.9\linewidth]{Images/SequenceDiagrams/SD_1.png}
                \caption{Student Registration Sequence Diagram}
                \label{fig: student_registration_seq_diag}
            \end{center}
        \end{figure}
        
\end{center}

\subsubsection*{UC\cuc . User Login}
\begin{center}
    \begin{longtable}{lp{0.75\linewidth}}
        \hline
        Actor            & Registered User                                                                                                                                                                               \\
        \hline
        Entry conditions & The User is registered on the \verb|CKB| platform and clicks the login button.                                                                                                               \\
        \hline
        Event Flow       
        & 1. The \verb|CKB| platform prompts the user to input institutional email.\\
        & 2. The \verb|CKB| platform validates the email redirecting the user to the institution's page.\\
        & 3. The \verb|CKB| platform communicates the outcome of the user's login.\\
        \hline
        Exit condition   & Login completed.   \\                                                                                                                                                                           
        \hline
        Exceptions   
        & 2.1. The user's email isn't valid.\\                                          
        & 2.2. The user's verification on the institution's page goes wrong. \\                                                                                                         
            & In these cases, the user receives a notification with an error message.   \\                                                               
        \hline
        \caption{User Login Use Case.}
        \label{tab: user_login_use_case}
    \end{longtable}

    %QUI SEQUENCE DIAGRAM
    \begin{figure} [H]
        \begin{center}
            \includegraphics[width=0.9\linewidth]{Images/SequenceDiagrams/SD_2.png}
            \caption{User Login Sequence Diagram}
            \label{fig: user_login_seq_diag}
        \end{center}
    \end{figure}
\end{center}


\subsubsection*{UC\cuc . Student Subscription to a Tournament}
\begin{center}
    \begin{longtable}{lp{0.75\linewidth}}
        \hline
        Actor            & Registered Student                                                                                                                                                                               \\
        \hline
        Entry conditions & The Student is logged in on the \verb|CKB| platform and clicks to the subscription button to a tournament.                                                                                                            \\
        \hline
        Event Flow       
        & 1. The \verb|CKB| platform receives the request.\\
        & 2. The \verb|CKB| platform checks if the tournament's subscription deadline is over.\\
        & 3. The \verb|CKB| platform communicates the outcome of the student's subscription.\\
        \hline
        Exit condition   & Subscription completed.   \\                                                                                                                                                                           
        \hline
        Exceptions   
        & 2.1. The tournament's subscription deadline is over.\\                                                                                                                                              
            & In these cases, the student receives a notification with an error message.   \\                                                               
        \hline
        \caption{Student Subscription Use Case.}
        \label{tab: student_subscription_use_case}
    \end{longtable}

    %QUI SEQUENCE DIAGRAM
    \begin{figure} [H]
        \begin{center}
            \includegraphics[width=0.9\linewidth]{Images/SequenceDiagrams/SD_3.png}
            \caption{Student Subscription Sequence Diagram}
            \label{fig: student_subscription_seq_diag}
        \end{center}
    \end{figure}
\end{center}

\subsubsection*{UC\cuc . Student creates a team}
\begin{center}
    \begin{longtable}{lp{0.75\linewidth}}
        \hline
        Actor            & Student                                                                                                                                                                               \\
        \hline
        Entry conditions & The student is logged in on the \verb|CKB| platform.                                                                                                               \\
        \hline
        Event Flow       
        & 1. The student clicks on the "Create team" button from the tournament page.\\
        & 2. The \verb|CKB| platform prompts the student to input the size of the team.\\
        & 3. The student inputs the size of the team.\\
        & 4. The \verb|CKB| platform prompts the student to input the name of the team.\\
        & 5. The student inputs the name of the team.\\
        & 6. The \verb|CKB| platform checks the name of the team.\\
        & 7. The \verb|CKB| platform communicates the outcome of the team creation.\\
        \hline
        Exit condition   & A team is created.   \\                                                                                                                                                                           
        \hline
        Exceptions   
        & 2.1. The team size is invalid.\\                                          
        & 4.1. The team name already exists or it is invalid. \\                                                                                                         
            & In these cases, the student receives a notification with an error message.   \\
        & 2-6.1. The student cancels team creation.\\
            & In this case, the \verb|CKB| platform will roll back any action performed and the student receives a notification with an update message.\\                                                               
        \hline
        \caption{Team Creation Use Case.}
        \label{tab: team_creation_use_case}
    \end{longtable}

    %QUI SEQUENCE DIAGRAM
    \begin{figure} [H]
        \begin{center}
            \includegraphics[width=0.9\linewidth]{Images/SequenceDiagrams/SD_4.png}
            \caption{Team Creation Sequence Diagram}
            \label{fig: team_creation_seq_diag}
        \end{center}
    \end{figure}
\end{center}


\subsubsection*{UC\cuc . Student invites another student to the team}
\begin{center}
    \begin{longtable}{lp{0.75\linewidth}}
        \hline
        Actor            & Student                                                                                                                                                                               \\
        \hline
        Entry conditions & The student is logged in on the \verb|CKB| platform and has already created a team.                                                                                                               \\
        \hline
        Event Flow       
        & 1. The student selects a student from the list of students on the tournament page.\\
        & 2. The student clicks on the invite button on the bottom of the list of students from the tournament page.\\
        & 3. The \verb|CKB| platform checks for remaining space in the team.\\
        & 4. The \verb|CKB| platform sends the invitation to the selected student.\\
        & 5. The \verb|CKB| platform communicates the outcome of the invitation to the student.\\
        \hline
        Exit condition   & An invitation is sent.   \\                                                                                                                                                                           
        \hline
        Exceptions   
        & 3.1. The team size is full.\\ 
        & 3.2. The student selected is already in another group.\\                                         
            & In these cases, the \verb|CKB| platform will notify the student with an error message.\\                                                               
        \hline
        \caption{Invitation Use Case.}
        \label{tab: invitation_use_case}
    \end{longtable}

    %QUI SEQUENCE DIAGRAM
    \begin{figure} [H]
        \begin{center}
            \includegraphics[width=0.9\linewidth]{Images/SequenceDiagrams/SD_5.png}
            \caption{Invitation Sequence Diagram}
            \label{fig: invitation_seq_diag}
        \end{center}
    \end{figure}
\end{center}

\subsubsection*{UC\cuc . Student accept an invitation to join a CKB team}
\begin{center}
    \begin{longtable}{lp{0.75\linewidth}}
        \hline
        Actor            & Student                                                                                                                                                                               \\
        \hline
        Entry conditions & The student is logged in on the \verb|CKB| platform and receives and invitation to join a CKB team.                                                                                                               \\
        \hline
        Event Flow       
        & 1. The student receives a pop-up norification informing of the invitation.\\
        & 2. The student clicks on the "Accept" button on the invitation pop-up.\\
        & 3. The \verb|CKB| platform sends the information to the invitation sender.\\
        & 4. The \verb|CKB| platform communicates the outcome of the invitation to both students.\\
        \hline
        Exit condition   & The two students are now part of the same team.\\                                                                                                                                                                           
        \hline
        Exceptions   
        & 3.1. The team is already dissembled.\\                                          
            & In this case, the \verb|CKB| platform will roll back any action performed and the student receives a notification with an update message.\\                                                               
        \hline
        \caption{Invitation Accepted Use Case.}
        \label{tab: invitation_accepted_use_case}
    \end{longtable}

    %QUI SEQUENCE DIAGRAM
    \begin{figure} [H]
        \begin{center}
            \includegraphics[width=0.9\linewidth]{Images/SequenceDiagrams/SD_6.png}
            \caption{Invitation Accepted Sequence Diagram}
            \label{fig: invitation_accepted_seq_diag}
        \end{center}
    \end{figure}
\end{center}

\subsubsection*{UC\cuc . Student rejects an invitation}
\begin{center}
    \begin{longtable}{lp{0.75\linewidth}}
        \hline
        Actor            & Student                                                                                                                                                                               \\
        \hline
        Entry conditions & The student is logged in on the \verb|CKB| platform and receives and invitation to join a CKB team.                                                                                                               \\
        \hline
        Event Flow       
        & 1. The student receives a pop-up norification of the invitation.\\
        & 2. The student clicks on the "Reject" button of the invitation pop-up.\\
        & 3. The \verb|CKB| platform sends the information to the invitation sender.\\
        & 4. The \verb|CKB| platform communicates the outcome of the invitation to both students.\\
        \hline
        Exit condition   & The students are not part of the same team.   \\                                                                                                                                                                           
        \hline
        \caption{Invitation Rejected Use Case.}
        \label{tab: invitation_rejected_use_case}
    \end{longtable}

    %QUI SEQUENCE DIAGRAM
    \begin{figure} [H]
        \begin{center}
            \includegraphics[width=0.9\linewidth]{Images/SequenceDiagrams/SD_7.png}
            \caption{Invitation Rejected Sequence Diagram}
            \label{fig: invitation_rejected_seq_diag}
        \end{center}
    \end{figure}
\end{center}


\subsubsection*{UC\cuc . Student joins a battle without a team}
\begin{center}
    \begin{longtable}{lp{0.75\linewidth}}
        \hline
        Actor            & Student                                                                                                                                                                               \\
        \hline
        Entry conditions & The Student is logged in on the \verb|CKB| platform and is registered to a tournament.                                                                                                            \\
        \hline
        Event Flow       
        & 1. The student selects a CKB from the list of battles in the Tournament page.\\
        & 2. The student clicks the "Join battle" button on the bottom of the list of battles in the tournament page.\\
        & 3. The \verb|CKB| platform verifies the registration deadline for the battle.\\
        & 4. The \verb|CKB| platform communicates the outcome of the student's action.\\
        \hline
        Exit condition   & The student joined the battle.   \\                                                                                                                                                                           
        \hline
        Exceptions   
        & 3.1. The battle's registration deadline is over.\\                                                                                                                                              
            & In this case, the student receives a notification with an error message.   \\                                                               
        \hline
        \caption{Student Join battle Use Case.}
        \label{tab: battle_alone_use_case}
    \end{longtable}

    %QUI SEQUENCE DIAGRAM
    \begin{figure} [H]
        \begin{center}
            \includegraphics[width=0.9\linewidth]{Images/SequenceDiagrams/SD_8.png}
            \caption{Student Join battle Sequence Diagram}
            \label{fig: battle_alone_seq_diag}
        \end{center}
    \end{figure}
\end{center}

\subsubsection*{UC\cuc . Student joins a battle with a team}
\begin{center}
    \begin{longtable}{lp{0.75\linewidth}}
        \hline
        Actor            & Team of students                                                                                                                                                                               \\
        \hline
        Entry conditions & The team is registered on the \verb|CKB| platform in a tournament's battle.                                                                                                            \\
        \hline
        Event Flow       
        & 1. The team leader selects a CKB from the list of battles in the Tournament page.\\
        & 2. The student clicks the "Join battle" button on the bottom of the list of battles in the tournament page.\\
        & 3. The \verb|CKB| platform verifies the registration deadline for the battle.\\
        & 4. The \verb|CKB| platform communicates the outcome of the student's action.\\
        \hline
        Exit condition   & The team joined the battle.   \\                                                                                                                                                                           
        \hline
        Exceptions   
        & 2.1. The number of students in the team is not in the range of the battle's minimum and maximum number of students.\\                                                                                                                                              
            & In this case, the student receives a notification with an error message and the team does not join the battle.   \\
        & 3.1. The battle's registration deadline is over.\\                                                                                                                                              
            & In this case, the student receives a notification with an error message.   \\                                                                
        \hline
        \caption{Team Join Battle Use Case.}
        \label{tab: team_join_battle_use_case}
    \end{longtable}

    %QUI SEQUENCE DIAGRAM
    \begin{figure} [H]
        \begin{center}
            \includegraphics[width=0.9\linewidth]{Images/SequenceDiagrams/SD_9.png}
            \caption{Team Join Battle Sequence Diagram}
            \label{fig: team_join_battle_seq_diag}
        \end{center}
    \end{figure}
\end{center}

\subsubsection*{UC\cuc . Educator creates a tournament}

\begin{center}
    \begin{longtable}{lp{0.75\linewidth}}
        \hline
        Actor            & Educator                                                                                                                                                                               \\
        \hline
        Entry conditions & The educator is logged in on the \verb|CKB| platform.\\                                                                                                               \\
        \hline
        Event Flow       
        & 1. The educator clicks on the "Create Tournament" button from the Menu in the homepage.\\
        & 2. The \verb|CKB| platform prompts the educator to input the tournament name.\\
        & 3. The educator inputs the tournament name.\\
        & 4. The \verb|CKB| platform prompts the educator to input the registration deadline.\\
        & 5. The educator inputs the registration deadline.\\
        & 6. The educator toggles the "Advanced Options" section.\\
        & 7. The \verb|CKB| platform prompts the educator to choose whether to include badges or not.\\
        & 8. The educator chooses whether to include badges or not.\\
        & 9. The \verb|CKB| platform prompts the educator to choose which badges to include.\\
        & 10. The educator chooses which badges to include.\\
        & 11. The \verb|CKB| platform prompts the educator to choose whether to create new badges or not.\\
        & 12. The educator chooses to create new badges.\\
        & 13. The \verb|CKB| platform opens a pop-up window allowing the educator to create new badges by selecting and combining variables and rules.\\
        & 14. The educator creates new badges.\\
        & 15. The educator pushes the "Create Tournament" button.\\
        & 16. The \verb|CKB| platform communicates the outcome of the tournament creation.\\
        & 17. The \verb|CKB| platform notifies all students that a new tournament is available.\\
        \hline
        Exit condition   & A tournament is created.   \\                                                                                                                                                                         
        \hline
        Exceptions   
        & 3.1. The educator does not input the tournament name.\\
            & The \verb|CKB| platform shows a message asking the user to input the tournament name.  \\
        & 5.1. The educator does not input the registration deadline or inputs an invalid date.\\
            & The \verb|CKB| platform shows a message asking the user to input the registration deadline.  \\
        & 8.1. The educator does not choose whether to include badges or not.\\
            & The \verb|CKB| platform assumes that badges are not included.  \\
        & 10.1. The educator does not choose which badges to include.\\
            & The \verb|CKB| platform assumes that no pre-existing badges are included.  \\
        & 12.1. The educator does not choose to create new badges.\\
            & The \verb|CKB| platform assumes that no new badges are created.  \\
        & 14.1. The educator tries to create a new badge without selecting any variable or rule.\\
            & The \verb|CKB| platform shows a message asking the user to select at least one variable or rule.  \\
        \hline
        \caption{Tournament Creation Use Case.}
        \label{tab: tournament_creation_use_case}
    \end{longtable}

    %QUI SEQUENCE DIAGRAM
    \begin{figure} [H]
        \begin{center}
            \includegraphics[width=0.9\linewidth]{Images/SequenceDiagrams/SD_10.png}
            \caption{Tournament Creation Sequence Diagram}
            \label{fig: tournament_creation_seq_diag}
        \end{center}
    \end{figure}
\end{center}


\subsubsection*{UC\cuc . Educator closes a tournament}
\begin{center}
    \begin{longtable}{lp{0.75\linewidth}}
        \hline
        Actor            & Educator \\
        \hline
        Entry conditions & The educator is logged in on the \verb|CKB| platform and has permissions to perform actions on a specific tournament.\\                                                                                                               \\
        \hline
        Event Flow       
        & 1. The educator clicks on a tournament for which he has editing permissions from the list in the "Tournaments" page\\
        & 2. The \verb|CKB| platform shows the tournament details page\\
        & 3. The educator clicks on the "Close Tournament" button\\
        & 4. The \verb|CKB| platform shows a pop-up window asking for confirmation\\
        & 5. The educator confirms the action\\
        & 6. The \verb|CKB| platform communicates the outcome of the tournament closing\\
        & 7. The \verb|CKB| computes the final ranking of the tournament and makes it available\\
        & 8. The \verb|CKB| platform notifies all students that the tournament is closed\\
        \hline
        Exit condition   & The tournament is closed.   \\                                                                                                                                                                         
        \hline
        Exceptions   
        & 5.1. The educator does not confirm the action\\
            & The \verb|CKB| platform does not close the tournament.  \\
        \hline
        \caption{Tournament Closing Use Case.}
        \label{tab: tournament_closing_use_case}
    \end{longtable}

    %QUI SEQUENCE DIAGRAM
    \begin{figure} [H]
        \begin{center}
            \includegraphics[width=0.9\linewidth]{Images/SequenceDiagrams/SD_11.png}
            \caption{Tournament Closing Sequence Diagram}
            \label{fig: tournament_closing_seq_diag}
        \end{center}
    \end{figure}
\end{center}

%. To create a new battle, an educator uses the CKB
%platform to perform the following steps:
%• upload the code kata (description and software project, including test cases and build
%automation scripts),
%• set minimum and maximum number of students per group,
%• set a registration deadline,
%• set a final submission deadline,
%• set additional configurations for scoring (see further details in the following).


\subsubsection*{UC\cuc . Educator creates a new Code Kata Battle}
\begin{center}
    \begin{longtable}{lp{0.75\linewidth}}
        \hline
        Actor            & Educator \\
        \hline
        Entry conditions & The educator is logged in on the \verb|CKB| platform and is on a tournament's details page. The educator has permissions to add new CKB to the tournament.\\                                                                                                               \\
        \hline
        Event Flow      
        & 1. The educator clicks on the "Create New Code Kata Battle" button\\ 
        & 2. The \verb|CKB| platform asks the educator to upload the code kata for the battle\\
        & 3. The educator uploads the code kata\\
        & 4. The \verb|CKB| platform verifies that the code kata includes a description and a software project with test cases and build automation scripts\\
        & 5. The \verb|CKB| platform asks the educator to set the minimum and maximum number of students per group\\
        & 6. The educator sets the minimum and maximum number of students per group\\
        & 7. The \verb|CKB| platform asks the educator to set a registration deadline\\
        & 8. The educator sets a registration deadline\\
        & 9. The \verb|CKB| platform asks the educator to set a final submission deadline\\
        & 10. The educator sets a final submission deadline\\
        & 11. The educator clicks on the "Additional Configurations for Scoring" button\\
        & 12. The \verb|CKB| platform shows the "Additional Configurations for Scoring" section\\
        & 13. The educator sets additional configurations for scoring if needed\\
        & 14. The educator clicks on the "Create" button\\
        & 15. The \verb|CKB| platform communicates the outcome of the CKB creation\\
        & 16. The \verb|CKB| platform notifies all students subscribed to the tournament that a new CKB is available\\
        \hline
        Exit condition   & The new CKB is added to the tournament.   \\                                                                                                                                                                         
        \hline
        Exceptions   
        & 4.1. The code kata does not include a description\\
            & The \verb|CKB| platform shows a message asking the educator to upload a code kata including a description.  \\
        & 4.2. The code kata does not include a software project\\
            & The \verb|CKB| platform shows a message asking the educator to upload a code kata including a software project.  \\
        & 4.3. The code kata does not include test cases\\
            & The \verb|CKB| platform shows a message asking the educator to upload a code kata including test cases.  \\
        & 4.4. The code kata does not include build automation scripts\\
            & The \verb|CKB| platform shows a message asking the educator to upload a code kata including build automation scripts.  \\
        & 5.1. The educator does not set the minimum and maximum number of students per group\\
            & The \verb|CKB| platform shows a message asking the educator to set the minimum and maximum number of students per group.  \\
        & 7.1. The educator does not set a registration deadline or sets an invalid date\\
            & The \verb|CKB| platform shows a message asking the educator to set a registration deadline.  \\
        & 9.1. The educator does not set a final submission deadline or sets an invalid date\\
            & The \verb|CKB| platform shows a message asking the educator to set a final submission deadline.  \\
        \hline
        \caption{New CKB creation Use Case.}
        \label{tab: new_CKB_use_case}
    \end{longtable}

    %QUI SEQUENCE DIAGRAM
    \begin{figure} [H]
        \begin{center}
            \includegraphics[width=0.9\linewidth]{Images/SequenceDiagrams/SD_12.png}
            \caption{New CKB creation Sequence Diagram}
            \label{fig: new_CKB_seq_diag}
        \end{center}
    \end{figure}
\end{center}


% Tournaments are created by an educator. Each battle created by an educator belongs to a specific tournament. Tournaments are created by an
%educator who can then grant to other colleagues the permission to create battles within the context of a specific tournament

\subsubsection*{UC\cuc . Educator grants another educator permissions to create battles in a tournament}
\begin{center}
    \begin{longtable}{lp{0.75\linewidth}}
        \hline
        Actor            & Educator \\
        \hline
        Entry conditions & The educator is logged in on the \verb|CKB| platform and is on a tournament's details page. The educator is the creator of the tournament.\\
        \hline
        Event Flow      
        & 1. The educator clicks on the "Add Administrator" button\\
        & 2. The \verb|CKB| platform asks the educator to input the email or username of the educator to whom grant permissions\\
        & 3. The educator inputs the email or username of the educator to whom grant permissions\\
        & 4. The \verb|CKB| platform communicates the outcome of the operation\\
        & 5. The \verb|CKB| platform notifies the educator to whom permissions have been granted\\
        \hline
        Exit condition   & The educator to whom permissions have been granted can create battles in the tournament.   \\
        \hline
        Exceptions   
        & 3.1. The educator inputs an invalid email or username or one belonging to a user who is not an educator\\
            & The \verb|CKB| platform shows a message asking the educator to input a valid email or username.  \\
        \hline
        \caption{Grant permission Use Case.}
        \label{tab: grant_permissions_use_case}
    \end{longtable}

    %QUI SEQUENCE DIAGRAM
    \begin{figure} [H]
        \begin{center}
            \includegraphics[width=0.9\linewidth]{Images/SequenceDiagrams/SD_13.png}
            \caption{Grant permission Sequence Diagram}
            \label{fig: grant_permissions_seq_diag}
        \end{center}
    \end{figure}
\end{center}


%The CKB platform also includes gamification badges: elements in the form of rewards that represent
%the achievements of individual students. Badges are defined by educators when they create a
%tournament. Each badge has a title (e.g., “top committer”) and one or more rules (i.e., simple Boolean properties) that must be fulfilled to achieve the badge. 
%Thus, each badge is assigned to one or more students depending on the rules checked at the end of the tournament. 
%The following are examples of possible badges:
%    • Title: tournament participant
%        o Rules: { tot_attended_battles > 0 }
%    • Title: top committer
%        o Rules: { tot_commits_student == max_tot_commits }
%Where tot_attended_battles, tot_commits_student and max_tot_commits are pre-defined variables
%that represent, respectively, the total number of battles the student has been involved in, the number
%of commits carried out by the student and the maximum total number of commits considering all
%students.
%As mentioned above, educators can create new badges and define new rules as well as new variables
%associated with them. Variables can represent any piece of information available in CKB relevant for
%scoring. It is up to you to identify such information and a simple language for creating new variables,
%rules, and badges.
        

\subsubsection*{UC\cuc . Educator creates a new badge}
\begin{center}
    \begin{longtable}{lp{0.75\linewidth}}
        \hline
        Actor            & Educator \\
        \hline
        Entry conditions & The educator is logged in on the \verb|CKB| platform and is on a tournament's details page. The educator is creating a tournament.\\
        \hline
        Event Flow      
        & 1. The educator clicks on the "Create New Badge" button in "Advanced Options"\\
        & 2. The \verb|CKB| platform prompts the educator to input the title of the badge\\
        & 3. The educator inputs the title of the badge\\
        & 4. The \verb|CKB| platform allows the educator to create new variables and rules for the badge\\
        & 5. The educator defines the rules of the badge through a wizard\\
        & 6. The educator clicks on the "Create" button\\
        & 7. The \verb|CKB| platform communicates the outcome of the badge creation\\
        \hline
        Exit condition   & The new badge is added to the tournament.   \\
        \hline
        Exceptions
        & 3.1. The educator does not input the title of the badge\\
            & The \verb|CKB| platform shows a message asking the educator to input the title of the badge.  \\
        & 5.1. The educator does not define any rule for the badge\\
            & The \verb|CKB| platform shows a message asking the educator to define at least one rule for the badge.  \\
        \hline
        \caption{New Badge creation Use Case.}
        \label{tab: badge_creation_use_case}
    \end{longtable}

    %QUI SEQUENCE DIAGRAM
    \begin{figure} [H]
        \begin{center}
            \includegraphics[width=0.9\linewidth]{Images/SequenceDiagrams/SD_14.png}
            \caption{New Badge creation Sequence Diagram}
            \label{fig: badge_creation_seq_diag}
        \end{center}
    \end{figure}
\end{center}


%To inject a gamified element, educators define badges to recognize and reward student achievements. 
%Badges, with titles like "Top Performer" or "Code Guru," serve as visual representations of accomplishments, enhancing student profiles and fostering a sense of achievement.
%Educators can include badges in tournaments by selecting from a list of pre-existing badges or creating new ones.
%When creating new badges, educators must define rules and variables that determine when a badge is awarded to a student.
%Rules and variables can either be selected from a list of pre-existing ones or created from scratch.
%In order to create a new variable, educators can make use of metrics exposed by the platform that offer information that could be relevant for scoring. 
%These metrics can be combined to arithmetic operations to create new variables.
%New rules can be defined by performing operation on variables (avreage, max, min, sum, etc.), resulting in complex requirements the students must meet to earn the badge.
\subsubsection*{UC\cuc . Educator creates a new variable and a new rule for a badge}
\begin{center}
    \begin{longtable}{lp{0.75\linewidth}}
        \hline
        Actor            & Educator \\
        \hline
        Entry conditions & The educator is logged in on the \verb|CKB| platform and is creating a new badge.\\
        \hline
        Event Flow      
        & 1. The educator clicks on the "Create New Variable" button at the end of the list of pre-existing variables\\
        & 2. The \verb|CKB| platform prompts the educator to input the name of the variable\\
        & 3. The educator inputs the name of the variable\\
        & 4. The platform shows a list of metrics exposed by CKB and available for the creation of the variable\\
        & 5. The educator selects one or more metrics from the list\\
        & 6. The educator writes an arithmetic expression (Spreadsheet formula) combining the selected metrics in the input box on the right of the list\\
        & 7. The educator clicks on the "Create" button\\
        & 8. The \verb|CKB| platform communicates the outcome of the variable creation\\
        & 9. The educator clicks on the "Create New Rule" button at the end of the list of pre-existing rules\\
        & 10. The \verb|CKB| platform prompts the educator to input the description of the rule\\
        & 11. The educator inputs the description of the rule\\
        & 12. The platform shows a list of variables available for the creation of the rule\\
        & 13. The educator writes a mathematical expression (Spreadsheet formula) combining the selected variables in the input box on the right of the list\\
        & 14. The educator clicks on the "Create" button\\
        & 15. The \verb|CKB| platform communicates the outcome of the rule creation\\
        \hline
        Exit condition   & The new variable and the new rule are added to the badge.   \\   
        \hline
        Exceptions
        & 6.1. The educator does not write a valid arithmetic expression\\
        & 13.1. The educator does not write a valid arithmetic expression\\
            & In both cases the \verb|CKB| platform shows a message asking the educator to write a valid arithmetic expression.  \\
        \caption{New Variable and Rule creation Use Case.}
        \label{tab: rule_variable_use_case}
    \end{longtable}

    %QUI SEQUENCE DIAGRAM
    \begin{figure} [H]
        \begin{center}
            \includegraphics[width=0.9\linewidth]{Images/SequenceDiagrams/SD_15.png}
            \caption{New Variable and Rule creation Sequence Diagram}
            \label{fig: rule_variable_seq_diag}
        \end{center}
    \end{figure}
\end{center}




% When the submission deadline of a CKB expires, there is a consolidation stage in
%which, if manual evaluation is required, the educator uses the CKB platform to go through the sources
%produced by each team to assign his/her score. Once the consolidation stage finishes, all students
%participating in the battle are notified when the final battle rank becomes available.
\subsubsection*{UC\cuc . Educator manually evaluates teams in a CKB}
\begin{center}
    \begin{longtable}{lp{0.75\linewidth}}
        \hline
        Actor            & Educator \\
        \hline
        Entry conditions & The educator is logged in on the \verb|CKB| platform and is on a CKB's details page. The educator is the creator of the CKB and the CKB has ended.\\
        \hline
        Event Flow      
        & 1. The educator clicks on the "Manually Evaluate Teams" button\\
        & 2. The \verb|CKB| platform shows the list of teams involved in the CKB\\
        & 3. The educator clicks on a team for which he wishes to perform a manual evaluation\\
        & 4. The \verb|CKB| platform shows the team's sources\\
        & 5. The educator assigns a score to the team\\
        & 6. The educator clicks on the "Save" button\\
        & 7. The \verb|CKB| platform communicates the outcome of the operation\\
        \hline
        Exit condition   & The score of the team is updated.   \\        
        \hline
        \caption{Manual evaluation Use Case.}
        \label{tab: manual_evaluation_use_case}
    \end{longtable}

    %QUI SEQUENCE DIAGRAM
    \begin{figure} [H]
        \begin{center}
            \includegraphics[width=0.9\linewidth]{Images/SequenceDiagrams/SD_16.png}
            \caption{Manual evaluation Sequence Diagram}
            \label{fig: manual_evaluation_seq_diag}
        \end{center}
    \end{figure}
\end{center}


% User clicks on a tournament from the list of tournaments in the homepage and in the details he can see the current ranking
\subsubsection*{UC\cuc . User visualizes a tournament's current ranking}
\begin{center}
    \begin{longtable}{lp{0.75\linewidth}}
        \hline
        Actor            & Educator/Student \\
        \hline
        Entry conditions & The user is logged in on the \verb|CKB| platform and is on the "Tournaments" page\\
        \hline
        Event Flow      
        & 1. The user clicks on a tournament from the list of tournaments in the "Tournaments" page\\
        & 2. The \verb|CKB| platform shows the tournament's details page\\
        & 3. The user clicks on the "Ranking" button\\
        & 4. The \verb|CKB| platform shows the tournament's current ranking\\
        \hline
        Exit condition   & The user visualizes the tournament's current ranking.   \\ 
        \hline
        \caption{Visualization of tournament's leaderboard.}
        \label{tab: visualization_of_leaderboard_use_case}
    \end{longtable}

    %QUI SEQUENCE DIAGRAM
    \begin{figure} [H]
        \begin{center}
            \includegraphics[width=0.9\linewidth]{Images/SequenceDiagrams/SD_17.png}
            \caption{Visualization of tournament's Diagram}
            \label{fig: visualization_of_leaderboard_seq_diag}
        \end{center}
    \end{figure}
\end{center}

%Badges can be visualized by all users. In particular, both students and educators can see collected
%badges when they visualize the profile of a student.
\subsubsection*{UC\cuc . User visualizes a student's profile}
\begin{center}
    \begin{longtable}{lp{0.75\linewidth}}
        \hline
        Actor            & Educator/Student \\
        \hline
        Entry conditions & The user is logged in on the \verb|CKB| platform and is browsing the "Users" page\\
        \hline
        Event Flow      
        & 1. The user clicks on a student from the list of students in the "Users" page\\
        & 2. The \verb|CKB| platform shows the student's profile page which includes the badges achieved by the student\\
        & 3. The user clicks on a badge\\
        & 4. The \verb|CKB| platform shows the badge's details\\
        \hline
        Exit condition   & The user visualizes the student's profile and the badges he earned.   \\ 
        \hline
        \caption{View Profile Use Case}
        \label{tab: view_profile_use_case}
    \end{longtable}

    %QUI SEQUENCE DIAGRAM
    \begin{figure} [H]
        \begin{center}
            \includegraphics[width=0.9\linewidth]{Images/SequenceDiagrams/SD_18.png}
            \caption{View Profile Sequence Diagram}
            \label{fig: view_profile_seq_diag}
        \end{center}
    \end{figure}
\end{center}


% The CKB platform automatically updates the battle score of a team as soon as new push actions on
% GitHub are performed. So, both students and educators involved in the battle can see the current rank evolving during the battle. 

\subsubsection*{UC\cuc . User visualizes a CKB's current ranking}
\begin{center}
    \begin{longtable}{lp{0.75\linewidth}}
        \hline
        Actor            & Educator/Student \\
        \hline
        Entry conditions & The user is logged in on the \verb|CKB| platform. The user is either an educator who administrates the CKB or a student who is participating in the CKB. \\
        \hline
        Event Flow      
        & 1. The user clicks on a tournament in which they are involved from the list of tournaments in the "Tournaments" page\\
        & 2. The \verb|CKB| platform shows the tournament's details page\\
        & 3. The user clicks on a CKB in which they are involved from the list of CKBs in the tournament's CKB list page\\
        & 4. The \verb|CKB| platform shows the CKB's details page\\
        & 5. The user clicks on the "Ranking" button\\
        & 6. The \verb|CKB| platform shows the CKB's current ranking\\
        \hline
        Exit condition   & The user visualizes the CKB's current ranking.   \\
        \hline
        \caption{CKB ranking Use Case}
        \label{tab: ckb_ranking_use_case}
    \end{longtable}

    %QUI SEQUENCE DIAGRAM
    \begin{figure} [H]
        \begin{center}
            \includegraphics[width=0.9\linewidth]{Images/SequenceDiagrams/SD_19.png}
            \caption{CKB ranking Sequence Diagram}
            \label{fig: ckb_ranking_seq_diag}
        \end{center}
    \end{figure}
\end{center}


\section{Performance Requirements}
\label{subsec:performance_requirements}%

\subsection*{1. Response Time:}
   - The \verb|CKB| platform shall respond to user interactions within an average time of 2 seconds, measured from the user action initiation to the completion of the corresponding operation.

\subsection*{2. Concurrent Users:}
   - The platform should support at least 1000 concurrent users without a significant degradation in response time.

\subsection*{3. GitHub Integration:}
   - GitHub repository creation and updates triggered by student commits shall be processed within 5 minutes of the GitHub action.

\subsection*{4. Automated Evaluation:}
   - Automated evaluation of submitted code shall be completed within 1 minute of each GitHub push action.

\subsection*{5. Scalability:}
   - The platform should accommodate at least a 20\% annual increase in user base and battle participation.

\subsection*{6. Data Retrieval Time:}
   - Information retrieval for ongoing tournaments, tournament ranks, and personal scores should be performed within an average time of 3 seconds.

\subsection*{7. Consolidation Stage:}
   - The consolidation stage, where manual evaluation is required, should be completed within 3 days after the submission deadline of a battle.

\subsection*{8. Badge Assignment:}
   - Badge assignment based on rules and variables should be executed within 1 day after the final tournament rank becomes available.

\subsection*{9. Platform Uptime:}
   - The platform should maintain a minimum of 99.9\% uptime over any given month.

\subsection*{10. Notification Latency:}
    - Notifications should be sent to all relevant users within 1 minute of the triggering event.

\subsection*{11. Gamification Features:}
    - Display of badges on user profiles and visualization of tournament-related achievements should be accessible within a response time of 3 seconds.

\subsection*{12. API Response Time:}
    - External APIs used for integrations should respond within 5 seconds.

\subsection*{13. Load Testing:}
    - The platform should undergo periodic load testing to handle peak loads without compromising performance.

\subsection*{14. Security Scan Duration:}
    - Security scans during code quality evaluation should not exceed 2 minutes.







\begin{comment}
\section{Performance Requirements}
\label{subsec:performance_requirements}%

\subsection*{1. Response Time:}
   - The CKB platform shall respond to user interactions (e.g., creating a battle, joining a battle, submitting code) within an average time of 2 seconds, measured from the user action initiation to the completion of the corresponding operation.

\subsection*{2. Concurrent Users:}
   - The platform should support at least 1000 concurrent users participating in different battles and tournaments without a significant degradation in response time.

\subsection*{3. GitHub Integration:}
   - GitHub repository creation and updates triggered by student commits shall be processed and reflected in the CKB platform within 5 minutes of the GitHub action.

\subsection*{4. Automated Evaluation:}
   - The automated evaluation of submitted code, including functional aspects, timeliness, and quality level, shall be completed within 1 minute of each GitHub push action.

\subsection*{5. Scalability:}
   - The CKB platform should be designed to handle a growth in the number of users, battles, and tournaments. It should accommodate at least a 20\% annual increase in user base and battle participation.

\subsection*{6. Data Retrieval Time:}
   - Information retrieval for ongoing tournaments, tournament ranks, and personal scores should be performed within an average time of 3 seconds.

\subsection*{7. Consolidation Stage:}
   - The consolidation stage, where manual evaluation is required, should be completed within 3 days after the submission deadline of a battle.

\subsection*{8. Badge Assignment:}
   - Badge assignment based on rules and variables should be executed within 1 day after the final tournament rank becomes available.

\subsection*{9. Platform Uptime:}
   - The CKB platform should maintain a minimum of 99.9\% uptime over any given month to ensure continuous availability for users.

\subsection*{10. Notification Latency:}
    - Notifications, such as battle creation, submission deadlines, and tournament closure, should be sent to all relevant users within 1 minute of the triggering event.

\subsection*{11. Gamification Features:}
    - The display of badges on user profiles and the visualization of tournament-related achievements should be accessible with a response time of 3 seconds.

\subsection*{12. API Response Time:}
    - Any external APIs used for integrations, such as GitHub Actions and static analysis tools, should respond within 5 seconds to avoid delays in platform functionalities.

\subsection*{13. Load Testing:}
    - The platform should undergo periodic load testing to ensure it can handle peak loads and unexpected spikes in user activity without compromising performance.

\subsection*{14. Security Scan Duration:}
    - The time required for security scans during the evaluation of code quality should not exceed 2 minutes.

\end{comment}


\section{Design Constraints}
\label{sec:design_constraints}%

\subsection{Standards Compliance}
\label{subsec:standards_compliance}%
The Code Kata Battle (CKB) platform must conform to established standards and legal requirements to ensure ethical and lawful operation. Key considerations include:

\begin{enumerate}
    \item \textbf{Privacy and Data Laws:}
          \begin{itemize}
              \item The platform must adhere to all laws governing privacy and data treatment, especially regarding user information and data exchange with third parties, such as CodeKata educators.
              \item Compliance with the European Union General Data Protection Regulation (EU GDPR) is mandatory to ensure the privacy and rights of users are protected.
          \end{itemize}

    \item \textbf{EU GDPR Principles:}
          \begin{itemize}
              \item The system design should align with the principles outlined in Article 5 of the GDPR document, emphasizing the lawful, fair, and transparent processing of personal data.
          \end{itemize}
\end{enumerate}

\subsection{Hardware Limitations}
\label{subsec:hardware_limitations}%
The design and functionality of the CKB platform need to consider specific limitations for optimal user experience:

\begin{enumerate}
    \item \textbf{Cross-Platform Accessibility:}
          \begin{itemize}
              \item The platform should be accessible from both web browsers and mobile applications. Mobile applications will be implemented for Android and iOS operating systems.
          \end{itemize}
\end{enumerate}
\textbf{Note:} Cross-platform accessibility, although listed under hardware limitations, primarily involves software compatibility rather than hardware constraints. It highlights the need for the platform to function seamlessly across different operating systems.

\subsection{Other Constraints}
While considering standard compliance and hardware limitations, additional constraints specific to the CKB project should be acknowledged:

\begin{enumerate}
    \item \textbf{Educator Configuration:}
          \begin{itemize}
              \item The design should facilitate easy configuration by educators, allowing them to customize scoring criteria, badge rules, and associated variables according to their preferences and requirements.
          \end{itemize}

    \item \textbf{Automated Testing Tools:}
          \begin{itemize}
              \item Seamless integration with automated testing tools is essential. The platform should be designed to accommodate any constraints or limitations imposed by these tools during code evaluation.
          \end{itemize}

    \item \textbf{Scalability:}
          \begin{itemize}
              \item The platform must be scalable to handle a growing user base, increasing numbers of battles, and additional features without compromising performance. Consideration should be given to both the user interface and backend infrastructure.
          \end{itemize}

    \item \textbf{Educator Workload:}
          \begin{itemize}
              \item The design should optimize the educator's workload, particularly during the manual evaluation stage. Streamlining the evaluation process will enhance efficiency and user satisfaction.
          \end{itemize}

    \item \textbf{Gamification Logic:}
          \begin{itemize}
              \item Constraints associated with the gamification logic, including badge creation, rule definition, and variable association, should be carefully considered to maintain system flexibility without compromising integrity.
          \end{itemize}

    \item \textbf{Integration with External Systems:}
          \begin{itemize}
              \item The platform needs to seamlessly integrate with external systems, such as GitHub, while considering any constraints or limitations imposed by these external systems on data exchange and interactions.
          \end{itemize}
\end{enumerate}

These design constraints are vital to guide the development of the Code Kata Battle platform, ensuring legal compliance, optimal user experience, and consideration of specific project-related constraints.




\section{Software System Attributes}
\label{sec:software_system_attributes}%

\subsection{Reliability}
\label{subsec:reliability}%
The reliability of the Code Kata Battle (CKB) platform is crucial for its users, especially during critical operations such as code evaluations and scoring. 
While the platform doesn't require absolute perfection in every operation, it must maintain a high level of reliability to instill confidence in educators and students. 
Considering the nature of the platform, a failure rate between $0.1\%$ and $1\%$ is deemed acceptable. 
This range positions the CKB system as a robust and dependable tool for code kata battles.

\subsection{Availability}
\label{subsec:availability}%
Availability is a key attribute for the CKB platform, particularly for educators who rely on its functionalities for managing tournaments, battles, and student assessments. 
Downtime, especially during crucial periods like ongoing battles or tournament deadlines, is unacceptable. Therefore, the platform aims to achieve a remarkable $99.9\%$ uptime, ensuring educators have uninterrupted access to critical features. 
This commitment to high availability enhances the overall user experience and reinforces the platform's reliability.

\subsection{Security}
\label{subsec:security}%
Security is paramount for the CKB platform, given its involvement in handling code submissions, communication with external systems like GitHub, and the storage of sensitive user information. 
The platform ensures data privacy by employing HTTPS for secure communication. Additionally, all stored information is encrypted, providing an extra layer of protection against potential security threats. 
By prioritizing these security measures, the CKB platform safeguards the integrity and confidentiality of user data, fostering trust among educators and students.

\subsection{Maintainability}
\label{subsec:maintainability}%
Maintainability is a critical aspect of the CKB platform's design philosophy. The platform is structured into modular components to facilitate efficient maintenance, updates, and future extensions. 
Each implemented functionality is meticulously documented, ensuring clarity for developers involved in maintenance tasks. 
The design emphasizes a modular approach, allowing updates to specific components without adversely affecting the entire system. 
This commitment to maintainability ensures that the CKB platform can adapt to evolving requirements and technologies without compromising its stability.

\subsection{Portability}
\label{subsec:portability}%
The CKB platform prioritizes portability to accommodate a diverse user base. Users can access the platform seamlessly from web browsers and mobile applications on both Android and iOS devices. 
The choice of programming languages and development tools is a strategic decision, with the goal of providing a consistent user experience across different platforms. 
While cross-platform development tools offer efficiency, separate implementations for Android and iOS are considered for a more tailored user experience. 
This flexibility in portability ensures that the CKB platform remains accessible and user-friendly, regardless of the chosen device or operating system.
