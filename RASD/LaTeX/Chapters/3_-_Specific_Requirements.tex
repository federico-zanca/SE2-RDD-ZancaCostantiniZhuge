\section{External Interface Requirements}
\label{sec:external_interface_requirements}%

\subsection{User Interfaces}
\label{subsec:user_interfaces}%
The \verb|CodeKataBattle (CKB)| platform is accessed via an intuitive and responsive web interface compatible with major browsers (Chrome, Firefox, Safari, etc.).
Educators enjoy a dedicated dashboard for effortless creation and management of tournaments, battles, and badges. This dashboard provides a comprehensive view of ongoing battles, tournament scores, and badge achievements.
Students utilize a user-friendly dashboard for team formation, battle participation, and progress tracking. It streamlines team formation, displays upcoming battles, current ranks, and summarizes earned badges.
GitHub seamlessly integrates into the platform for code versioning and automated testing. Students can easily fork repositories, set up GitHub Actions, and monitor build and test results within the \verb|CKB| platform.
To keep all stakeholders informed, the platform employs a robust notification system. This system supports both email notifications and in-platform alerts, 
ensuring timely updates for educators and students on critical events like upcoming deadlines or changes in battle status.

\subsection{Hardware Interfaces}
\label{subsec:hardware_interfaces}%
The \verb|CKB| platform prioritizes accessibility by ensuring compatibility across a diverse range of devices. 
Users can seamlessly access the platform from desktop computers, laptops, tablets, and smartphones. This inclusivity caters to the varied preferences and device usage patterns of our user base. 
The platform's responsive design ensures that the user interface adapts fluidly to different screen sizes, providing an optimal experience regardless of the device used. 
This commitment to device compatibility aims to enhance user convenience and flexibility, promoting a versatile and user-centric engagement with the \verb|CKB| platform.

\subsection{Software Interfaces}
\label{subsec:software_interfaces}%
The \verb|CKB| platform seamlessly communicates with GitHub via APIs, enabling functionalities such as repository creation, commit tracking, and automated test processes.
To ensure a secure integration, the platform should smoothly connect with GitHub APIs, facilitating automated workflows triggered by student commits.
Additionally, the platform harnesses static analysis tools to assess code quality comprehensively.
Incorporating these tools seamlessly, educators can tailor automated evaluations by configuring specific aspects like security, reliability, and maintainability. This flexibility ensures a nuanced understanding of the code's overall quality.

\subsection{Communication Interfaces}
\label{subsec:communication_interfaces}%
The platform actively engages in communication with students, delivering notifications, battle updates, and final results in a secure manner through HTTPS. A reliable messaging system ensures timely information dissemination to students.
Similarly, educators stay well-informed through the platform, receiving notifications and updates on battle progress and final results. Secure communication channels, similar to those used for students, guarantee the confidentiality and reliability of information relayed to educators.
For the configuration of badges and rules, educators seamlessly use the platform to define new badges, rules, and associated variables. Ensuring a user-friendly interface, educators can make real-time adjustments, with changes promptly reflecting across the platform. This intuitive process empowers educators to tailor the platform to their evolving requirements effortlessly.

\section{Functional Requirements}
\label{sec:functional_requirements}%

\subsection{Requirements}
\label{subsec: requirements}%
The \verb|CKB| platform offers several functionalities to both educators and students.
In the following table they are listed all the detected requirements that the platform should respect in order to guarantee
the satisfiability of the goals:
\newpage
\newcounter{req}
\setcounter{req}{1}
\newcommand{\creq}{\thereq\stepcounter{req}}
\textbf{Educators}
\begin{center}
    \begin{longtable}{|l|p{0.9\linewidth}|}
        \hline
        \textbf{ID} & \textbf{Description}                                                                                                                             \\
        \hline
        R\creq      & The \verb|CKB| platform shall allow educators to create an account.                                                                    \\
        \hline
        R\creq      & The \verb|CKB| platform shall allow educators to log in.                                                                                 \\
        \hline
        R\creq      & The \verb|CKB| platform shall allow educators to create a new tournament.                                                                \\
        \hline
        R\creq      & The \verb|CKB| platform shall allow educators to set the minimum and maximum number of students per group for a tournament.                                                        \\
        \hline
        R\creq      & The \verb|CKB| platform shall allow educators to upload a code kata battle.                                                                         \\
        \hline
        R\creq      & The \verb|CKB| platform shall allow educators to grant permissions to other educators to create battles within a specific tournament.                                                      \\
        \hline
        R\creq      & The \verb|CKB| platform shall enable educators to include a battle to a specific tournament.                                                      \\
        \hline
        R\creq      & The \verb|CKB| platform shall allow educators to include a description for a battle.                                                                \\
        \hline
        R\creq      & The \verb|CKB| platform shall allow educators to include a software project with test cases.                                                                \\
        \hline
        R\creq      & The \verb|CKB| platform shall allow educators to set a registration deadline for a battle within a tournament.                                          \\
        \hline
        R\creq      & The \verb|CKB| platform shall allow educators to set a final submission deadline for a battle within a tournament.                                                                  \\
        \hline
        R\creq      & The \verb|CKB| platform shall allow educators to set additional configurations for scoring, including functional aspects and quality level criteria.                               \\
        \hline
        R\creq      & The \verb|CKB| platform shall allow educators to close a tournament.                                                  \\
        \hline
        R\creq      & The \verb|CKB| platform shall allow an educator to define optional manual evaluation criteria for score assignment in battles.   \\
        \hline
        R\creq      & The \verb|CKB| platform shall allow educators to manually evaluate and assign scores to teams.                                           \\
        \hline
        R\creq      & The \verb|CKB| platform shall allow educators and students to visualize the gamification badges.                                                      \\
        \hline
        R\creq      & The \verb|CKB| platform shall allow educators to define new badges for gamification.                                                             \\
        \hline
        R\creq      & The \verb|CKB| platform shall allow educators to define new rules associated with the badges.                                                      \\
        \hline
        R\creq      & The \verb|CKB| platform shall allow educators to define new variables associated with the badges.                                                      \\
        \hline
        R\creq      & The \verb|CKB| platform shall allow all students and educators to see the ranking of each ongoing tournament with the score of each student subscribed.\\
        \hline
        \caption{Educator Requirements.}
        \label{tab: req}%
    \end{longtable}
\end{center}

\textbf{Students}
\begin{center}
    \begin{longtable}{|l|p{0.9\linewidth}|}
        \hline
        \textbf{ID} & \textbf{Description}                                                                                                                             \\
        \hline
        R\creq      & The \verb|CKB| platform shall not allow students to participate a tournament after the registration deadline.                                                      \\
        \hline
        R\creq      & The \verb|CKB| platform shall not allow students to participate a battle after the registration deadline.                                                      \\
        \hline
        R\creq      & The \verb|CKB| platform shall allow students to subscribe to a tournament within a specified deadline.                                                  \\
        \hline
        R\creq      & The \verb|CKB| platform shall allow students to create teams for a specific battle within a tournament.                                                      \\
        \hline
        R\creq      & The \verb|CKB| platform shall allow students to join teams for a specific battle within a tournament.                                                      \\
        \hline
        R\creq      & The \verb|CKB| platform shall allow students to invite other participants to the same group.                               \\
        \hline
        R\creq      & The \verb|CKB| platform shall allow students to accept an invitation.                               \\
        \hline
        R\creq      & The \verb|CKB| platform shall allow students to reject an invitation.                               \\
        \hline
        R\creq      & The \verb|CKB| platform shall allow students to join a battle without a team.                               \\
        \hline
        R\creq      & The \verb|CKB| platform shall allow students to upload a solution for a specific battle on behalf of the team.                                                      \\
        \hline
        \caption{Student Requirements.}
        \label{tab: req}%
    \end{longtable}
\end{center}

\textbf{Platform}
\begin{center}
    \begin{longtable}{|l|p{0.9\linewidth}|}
        \hline
        \textbf{ID} & \textbf{Description}                                                                                                                             \\
        \hline
        R\creq      & The \verb|CKB| platform shall notify all subscribed students of a new battle and its details within a specific tournament.                               \\
        \hline
        R\creq      & The \verb|CKB| platform shall notify all subscribed students of a new tournament and its details.                               \\
        \hline
        R\creq      & The \verb|CKB| platform shall create a GitHub repository for each battle.                                        \\
        \hline
        R\creq      & send a link to the Github repository associated to a battle to all members of subscribed teams upon expiration of the registration deadline. \\
        \hline
        R\creq      & The \verb|CKB| platform shall be able to be informed of new students' commits by Github Actions workflows. \\
        \hline
        R\creq      & The \verb|CKB| platform shall be able to pull the latest sources from the forks of the Github repository provided. \\
        \hline
        R\creq      & The \verb|CKB| platform shall be able to run the testcases on the code uploaded by students and determine if the code is a valid solution for the exercise.\\
        \hline
        R\creq      & The \verb|CKB| platform shall inform students of the mandatory automated evaluation criteria, including functional aspects, timeliness, and source code quality.                                        \\
        \hline
        R\creq      & The \verb|CKB| platform shall automatically update the battle score of a team based on GitHub commits and test results.                                   \\
        \hline
        R\creq      & The \verb|CKB| platform shall automatically close a finished battle.                                                      \\
        \hline
        R\creq      & The \verb|CKB| platform shall assign or update battle scores to each team of the battle.                                                      \\
        \hline
        R\creq      & The \verb|CKB| platform shall calculate and update the personal tournament score of each student based on their performance in battles.                    \\
        \hline
        R\creq      & The \verb|CKB| platform shall be able to create a ranking of teams for every tournament.                                                      \\
        \hline
        R\creq      & The \verb|CKB| platform shall keep track of time elapsed from the start of a CKB and the final submissions of each team. \\
        \hline
        R\creq      & The \verb|CKB| platform shall be able to use static analysis tools to evaluate the quality of the code submitted by teams in terms of security, reliability, maintainability and other aspects defined by the educator who created the battle. \\
        \hline
        R\creq      & The \verb|CKB| platform shall notify all students involved in a tournament when it is closed and the final ranking is available.                                                       \\
        \hline
        R\creq      & The \verb|CKB| platform shall visualize ongoing tournaments and their ranks for all users.                                                                 \\
        \hline
        R\creq      & The \verb|CKB| platform shall display collected badges on the profile of both students and educators.                                                                    \\
        \hline
        \caption{Platform Requirements.}
        \label{tab: req}%
    \end{longtable}
\end{center}

\subsection{Mapping on goals}
\label{subsec: map_on_g}%
In the following section it is shown how the relation $R\land D \models G$ holds.
In particular, at first it is shown a traceability matrix that associates domain assumptions and requirements to each goal.
After that, to facilitate reading, the section reports the text of all the assumptions and all the requirements related to each goal.
\newcounter{mg}
\setcounter{mg}{1}
\newcommand{\cmg}{\themg\stepcounter{mg}}
\begin{center}
    \begin{longtable}{|p{0.06\linewidth}|p{0.34\linewidth}|p{0.6\linewidth}|}
        \hline
        \textbf{Goal} & \textbf{Domain assumptions}                       & \textbf{Requirements}                                                               \\
        \hline
        G\cmg         &                             & R23, R24, R25, R29, R30                        \\
        \hline
        G\cmg         &                             & R24, R25, R26, R27, R28                                      \\
        \hline
        G\cmg         &                             & R1, R2, R3, R4, R5, R6, R7, R8, R9, R10, R11, R12, R13, R14, R15, R16, R17, R18, R19, R20                                \\
        \hline
        G\cmg         &                             & R33, R34, R35, R36, R37, R38, R39              \\
        \hline
        G\cmg         &                             & R20, R29, R30, R44                               \\
        \hline
        G\cmg         &                             & R41, R42, R45                     \\
        \hline
        G\cmg         &                             & R46, R47                                \\
        \hline
        G\cmg         &                             & R48                                                   \\
        \hline
        G\cmg         &                             & R16, R17, R18, R19, R48                                               \\
        \hline
        \caption{Mapping on goals.}
        \label{tab: map_on_g}%
    \end{longtable}
\end{center}

In this section, it will be shown the functional requirements and the domain assumption related to each goal.
\begin{itemize}

    \item \textbf{{[G.1]} Enable students to enhance their software development skills through coding challenges.}
    \begin{itemize}
        \item \textbf{[R.23]} The \verb|CKB| platform shall allow students to subscribe to a tournament within a specified deadline.
        \item \textbf{[R.24]} The \verb|CKB| platform shall allow students to create teams for a specific battle within a tournament.
        \item \textbf{[R.25]} The \verb|CKB| platform shall allow students to join teams for a specific battle within a tournament. 
        \item \textbf{[R.29]} The \verb|CKB| platform shall allow students to join a battle without a team. 
        \item \textbf{[R.30]} The \verb|CKB| platform shall allow students to upload a solution for a specific battle on behalf of the team.
        \item \textbf{[D.1]}
    \end{itemize}

    \item \textbf{{[G.2]} Facilitate peer learning and collaboration by allowing students to form teams and work together on coding exercises. }
        \begin{itemize}
            \item \textbf{[R.24]} The \verb|CKB| platform shall allow students to create teams for a specific battle within a tournament.
            \item \textbf{[R.25]} The \verb|CKB| platform shall allow students to join teams for a specific battle within a tournament.
            \item \textbf{[R.26]} The \verb|CKB| platform shall allow students to invite other participants to the same group. 
            \item \textbf{[R.27]} The \verb|CKB| platform shall allow students to join a battle without a team.
            \item \textbf{[R.28]} The \verb|CKB| platform shall allow students to reject an invitation.
            \item \textbf{[D.1]}
        \end{itemize}

        \item \textbf{{[G.3]} Provide educators with a platform to create and manage coding challenges tailored to specific learning objectives. }
        \begin{itemize}
            \item \textbf{[R.1]} The \verb|CKB| platform shall allow educators to create an account.
            \item \textbf{[R.2]} The \verb|CKB| platform shall allow educators to log in. 
            \item \textbf{[R.3]} The \verb|CKB| platform shall allow educators to create a new tournament.
            \item \textbf{[R.4]} The \verb|CKB| platform shall allow educators to set the minimum and maximum number of students per group for a tournament.
            \item \textbf{[R.5]} The \verb|CKB| platform shall allow educators to upload a code kata battle.
            \item \textbf{[R.6]} The \verb|CKB| platform shall allow educators to grant permissions to other educators to create battles within a specific tournament.
            \item \textbf{[R.7]} The \verb|CKB| platform shall enable educators to include a battle to a specific tournament. 
            \item \textbf{[R.8]} The \verb|CKB| platform shall allow educators to include a description for a battle.
            \item \textbf{[R.9]} The \verb|CKB| platform shall allow educators to include a software project with test cases.  
            \item \textbf{[R.10]} The \verb|CKB| platform shall allow educators to set a registration deadline for a battle within a tournament.
            \item \textbf{[R.11]} The \verb|CKB| platform shall allow educators to set a final submission deadline for a battle within a tournament.
            \item \textbf{[R.12]} The \verb|CKB| platform shall allow educators to set additional configurations for scoring, including functional aspects and quality level criteria.
            \item \textbf{[R.13]} The \verb|CKB| platform shall allow educators to close a tournament. 
            \item \textbf{[R.14]} The \verb|CKB| platform shall allow an educator to define optional manual evaluation criteria for score assignment in battles.
            \item \textbf{[R.15]} The \verb|CKB| platform shall allow educators to manually evaluate and assign scores to teams. 
            \item \textbf{[R.16]} The \verb|CKB| platform shall allow educators and students to visualize the gamification badges.
            \item \textbf{[R.17]} The \verb|CKB| platform shall allow educators to define new badges for gamification.
            \item \textbf{[R.18]} The \verb|CKB| platform shall allow educators to define new rules associated with the badges.
            \item \textbf{[R.19]} The \verb|CKB| platform shall allow educators to define new variables associated with the badges.
            \item \textbf{[R.20]} The \verb|CKB| platform shall allow all students and educators to see the ranking of each ongoing tournament with the score of each student subscribed.
        \end{itemize}

        \item \textbf{{[G.4]} Implement automated evaluation of code submissions, considering factors such as functional correctness, timeliness, and code quality. }
        \begin{itemize}
            \item \textbf{[R.33]} The \verb|CKB| platform shall create a GitHub repository for each battle.
            \item \textbf{[R.34]} send a link to the Github repository associated to a battle to all members of subscribed teams upon expiration of the registration deadline.
            \item \textbf{[R.35]} The \verb|CKB| platform shall be able to be informed of new students' commits by Github Actions workflows. 
            \item \textbf{[R.36]} The \verb|CKB| platform shall be able to pull the latest sources from the forks of the Github repository provided.
            \item \textbf{[R.37]} The \verb|CKB| platform shall be able to run the testcases on the code uploaded by students and determine if the code is a valid solution for the exercise.
            \item \textbf{[R.38]} The \verb|CKB| platform shall inform students of the mandatory automated evaluation criteria, including functional aspects, timeliness, and source code quality.  
            \item \textbf{[R.39]} The \verb|CKB| platform shall automatically update the battle score of a team based on GitHub commits and test results.
        \end{itemize}

        \item \textbf{{[G.5]} Introduce a competitive element through tournament-based coding challenges, motivating students to strive for excellence. }
        \begin{itemize}
            \item \textbf{[R.20]} The \verb|CKB| platform shall allow all students and educators to see the ranking of each ongoing tournament with the score of each student subscribed.
            \item \textbf{[R.29]} The \verb|CKB| platform shall allow students to join a battle without a team.
            \item \textbf{[R.30]} The \verb|CKB| platform shall allow students to upload a solution for a specific battle on behalf of the team. 
            \item \textbf{[R.44]} The \verb|CKB| platform shall keep track of time elapsed from the start of a CKB and the final submissions of each team
        \end{itemize}

        \item \textbf{{[G.6]} Assign scores to teams based on automated and manual evaluation, creating a transparent and fair ranking system. }
        \begin{itemize}
            \item \textbf{[R.41]} The \verb|CKB| platform shall assign or update battle scores to each team of the battle.
            \item \textbf{[R.42]} The \verb|CKB| platform shall calculate and update the personal tournament score of each student based on their performance in battles.
            \item \textbf{[R.45]} The \verb|CKB| platform shall be able to use static analysis tools to evaluate the quality of the code submitted by teams in terms of security, reliability, maintainability and other aspects defined by the educator who created the battle
        \end{itemize}

        \item \textbf{{[G.7]} Notify participants when a tournament concludes, providing closure and allowing reflection on performance. }
        \begin{itemize}
            \item \textbf{[R.46]} The \verb|CKB| platform shall notify all students involved in a tournament when it is closed and the final ranking is available.   
            \item \textbf{[R.47]} The \verb|CKB| platform shall visualize ongoing tournaments and their ranks for all users. 
        \end{itemize}

        \item \textbf{{[G.8]} Educators can close tournaments, and the platform notifies all students when the final tournament rank becomes available. }
        \begin{itemize}
            \item \textbf{[R.48]} The \verb|CKB| platform shall display collected badges on the profile of both students and educators.   
        \end{itemize}

        \item \textbf{{[G.9]} Students and educators seek information about ongoing tournaments, ranks, and profiles with badges.}
        \begin{itemize}
            \item \textbf{[R.16]} The \verb|CKB| platform shall allow educators and students to visualize the gamification badges.
            \item \textbf{[R.17]} The \verb|CKB| platform shall allow educators to define new badges for gamification.
            \item \textbf{[R.18]} The \verb|CKB| platform shall allow educators to define new rules associated with the badges.
            \item \textbf{[R.19]} The \verb|CKB| platform shall allow educators to define new variables associated with the badges.
            \item \textbf{[R.48]} The \verb|CKB| platform shall display collected badges on the profile of both students and educators.   
            \item \textbf{[D.1]}
        \end{itemize}
\end{itemize}


\section{Performance Requirements}
\label{subsec:performance_requirements}%

\subsection*{1. Response Time:}
   - The \verb|CKB| platform shall respond to user interactions within an average time of 2 seconds, measured from the user action initiation to the completion of the corresponding operation.

\subsection*{2. Concurrent Users:}
   - The platform should support at least 1000 concurrent users without a significant degradation in response time.

\subsection*{3. GitHub Integration:}
   - GitHub repository creation and updates triggered by student commits shall be processed within 5 minutes of the GitHub action.

\subsection*{4. Automated Evaluation:}
   - Automated evaluation of submitted code shall be completed within 1 minute of each GitHub push action.

\subsection*{5. Scalability:}
   - The platform should accommodate at least a 20\% annual increase in user base and battle participation.

\subsection*{6. Data Retrieval Time:}
   - Information retrieval for ongoing tournaments, tournament ranks, and personal scores should be performed within an average time of 3 seconds.

\subsection*{7. Consolidation Stage:}
   - The consolidation stage, where manual evaluation is required, should be completed within 3 days after the submission deadline of a battle.

\subsection*{8. Badge Assignment:}
   - Badge assignment based on rules and variables should be executed within 1 day after the final tournament rank becomes available.

\subsection*{9. Platform Uptime:}
   - The platform should maintain a minimum of 99.9\% uptime over any given month.

\subsection*{10. Notification Latency:}
    - Notifications should be sent to all relevant users within 1 minute of the triggering event.

\subsection*{11. Gamification Features:}
    - Display of badges on user profiles and visualization of tournament-related achievements should be accessible within a response time of 3 seconds.

\subsection*{12. API Response Time:}
    - External APIs used for integrations should respond within 5 seconds.

\subsection*{13. Load Testing:}
    - The platform should undergo periodic load testing to handle peak loads without compromising performance.

\subsection*{14. Security Scan Duration:}
    - Security scans during code quality evaluation should not exceed 2 minutes.







\begin{comment}
\section{Performance Requirements}
\label{subsec:performance_requirements}%

\subsection*{1. Response Time:}
   - The CKB platform shall respond to user interactions (e.g., creating a battle, joining a battle, submitting code) within an average time of 2 seconds, measured from the user action initiation to the completion of the corresponding operation.

\subsection*{2. Concurrent Users:}
   - The platform should support at least 1000 concurrent users participating in different battles and tournaments without a significant degradation in response time.

\subsection*{3. GitHub Integration:}
   - GitHub repository creation and updates triggered by student commits shall be processed and reflected in the CKB platform within 5 minutes of the GitHub action.

\subsection*{4. Automated Evaluation:}
   - The automated evaluation of submitted code, including functional aspects, timeliness, and quality level, shall be completed within 1 minute of each GitHub push action.

\subsection*{5. Scalability:}
   - The CKB platform should be designed to handle a growth in the number of users, battles, and tournaments. It should accommodate at least a 20\% annual increase in user base and battle participation.

\subsection*{6. Data Retrieval Time:}
   - Information retrieval for ongoing tournaments, tournament ranks, and personal scores should be performed within an average time of 3 seconds.

\subsection*{7. Consolidation Stage:}
   - The consolidation stage, where manual evaluation is required, should be completed within 3 days after the submission deadline of a battle.

\subsection*{8. Badge Assignment:}
   - Badge assignment based on rules and variables should be executed within 1 day after the final tournament rank becomes available.

\subsection*{9. Platform Uptime:}
   - The CKB platform should maintain a minimum of 99.9\% uptime over any given month to ensure continuous availability for users.

\subsection*{10. Notification Latency:}
    - Notifications, such as battle creation, submission deadlines, and tournament closure, should be sent to all relevant users within 1 minute of the triggering event.

\subsection*{11. Gamification Features:}
    - The display of badges on user profiles and the visualization of tournament-related achievements should be accessible with a response time of 3 seconds.

\subsection*{12. API Response Time:}
    - Any external APIs used for integrations, such as GitHub Actions and static analysis tools, should respond within 5 seconds to avoid delays in platform functionalities.

\subsection*{13. Load Testing:}
    - The platform should undergo periodic load testing to ensure it can handle peak loads and unexpected spikes in user activity without compromising performance.

\subsection*{14. Security Scan Duration:}
    - The time required for security scans during the evaluation of code quality should not exceed 2 minutes.

\end{comment}