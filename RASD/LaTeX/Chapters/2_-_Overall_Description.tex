\section{Product perspective}
\label{sec:product_perspective}%

\subsection{Class Diagrams}
\label{subsec:class_diagrams}%
The diagram below represents and describes the classes involved in the system, their basic functionalities, attributes,
and the relationships between them.
From the diagram, it is easy to see how the tournament entity function as a bridge between different
components. More specifically, in our design we assigned to this entity the task to hold track 
of battles, leaderboard of the tournament itself and badge assignment.
All the other functionalities are esily understandable from within the diagram or 
will be later discussed. \newline
Here, we want to stress that, even though just an high level view of the system-to-be, some consideration of implementation can be done. 
More specifically, some design pattern can be taken into consideration:
\begin{itemize}
    \item Decorator Pattern: to implement the logic behind the scoring system. Educator's choice of different aspect to consider will be easily managable 
    \item Observer patter: to implement the structure behind the notification system.
    \item Factory pattern: for the creation of different Badges with different characteristics and requirements.
  \end{itemize}

\begin{figure}[H]
    \begin{center}
        \includegraphics[width=0.9\linewidth]{Images/class-diagram.jpeg}
        \caption{A simplified Class Diagram}
        \label{fig:class_diagram_pic}%
    \end{center}
\end{figure}

\subsection{Scenarios}
\label{subsec:scenarios}%

\paragraph{Unregistered student creates an account.}
Gianmarco is a computer science student from Politecnico of Milan. After attending several courses about software engineering, 
he is looking for a way to practice his skills. Fortunately, one of his professor suggests him the CodeKataBattle platform.
Gianmarco immediately proceeds to create an account. He navigates to the platform and goes to the "sign up" section. 
The system asks Gianmarco an email linked 
to an institutional profile valid for the CKB platform (for example name.surname@mail.polimi.it). He is then redirected to the log in page of his institution.  
After a successful authentication, Gianmarco will be prompted with a data access request page for sharing data about his university profile. This way the CKB platform 
can determine wheter Gianmarco is a student or a professor and act accordingly. 
Once accepted the data access request, he is redirected to the creation profile page, where he is asked to insert an unique username and a password compliant to security policy of the system. 
After accepting the terms \& conditions 
and submitting everything, the system creates his account, if info are correct, and Gianmarco can begin to use the platform.

\paragraph*{User Logs in.}
Emily, the coding enthusiast, is all set for her daily dose of challenges on the CodeKataBattle (CKB) platform. 
To embark on her coding adventure, she clicks the login button and enters her institutional email.
The platform asks Emily to provide the password for her account. 
Excited and confident, Emily types in her password, which is verified by the platform to ensure it matches the email/password combination for the user.
With a successful password match, the platform welcomes Emily to her homepage; she's now ready to dive into the coding battles.
In case there's a typo in her email or a hiccup during password verification, the platform gently guides Emily through the process, ensuring a seamless login experience.

%Actor Student
%Entry conditions The student is logged in on the CKB platform.
%Event Flow 1. The student clicks on the "Create team" button from the tour-
%nament page.
%2. The CKB platform prompts the student to input the size of the
%team.
%3. The student inputs the size of the team.
%4. The CKB platform prompts the student to input the name of the
%team.
%5. The student inputs the name of the team.
%6. The CKB platform checks the name of the team.
%7. The CKB platform communicates the outcome of the team cre-
%ation.
%Exit condition A team is created.
\paragraph*{Student creates a team.}
Francesca, a student on the CodeKataBattle (CKB) platform, is eager to participate in the "Algorithmic Challenge" CKB.
Despite the possibility to participate individually to this battle, 
Francesca is aware that winning it all by herself is a daunting task, so she decides to form a team.
She navigates to the tournament details page and clicks on the "Create Team" button and is prompted to input the size of the team compliant to the settings of the battle (min/max number of students per team).
Francesca wants to form a team of three students, so she chooses the number three from the drop-down menu. Moreover, she decides to make her team public, so that other students can aks to join the team.
After deciding that the name of her team will be "The Three Musketeers," she clicks on the "Create Team" button.
She will now be able to send other students team invitation and accpete or decline joining requests.


%Actor Student
%Entry conditions The student is logged in on the CKB platform and created a team.
%Event Flow 1. The student selects a student from the list of students on the
%tournament page.
%2. The student clicks on the invite button on the bottom of the list
%of students from the tournament page.
%3. The CKB platform checks for remaining space in the team.
%4. The CKB platform sends the invitation to the selected student.
%5. The CKB platform communicates the outcome of the invitation
%to the student.
%Exit condition An invitation is sent.
\paragraph*{Student invites another student to join a team.}
Imagine Marco, an enthusiastic student navigating the CodeKataBattle (CKB) platform, keen on forming a proficient coding team for an upcoming battle. 
Having already gathered part of his team, Marco identifies Sofia, another student he believes would enhance their collective skills.
On the tournament page, Marco locates Sofia's profile in the lists of students not in a team yet and clicks the "Invite" button, signaling his desire to collaborate. 
The CKB platform efficiently checks team capacity, ensuring seamless integration.
In the digital realm, an invitation is generated and dispatched to Sofia through the platform's communication channels. 
Marco eagerly awaits the platform's response, hopeful for a positive outcome.
Success! The invitation is sent, propelling Marco's team towards a cohesive coding alliance.



\paragraph*{Student accepts an invitation to join a team.}
Sofia, a student passionate about coding, is actively using the CKB platform to participate in coding tournaments. 
Today, as she's logged in, he receives a pop-up notification indicating that she has been invited by Marco to join a team for a CKB.
Intrigued, she clicks on the "Accept" button within the invitation pop-up as she knows that her and Marco would be a good team. 
Behind the scenes, the CKB platform swiftly processes this action.
If the team didn't fill up in the meantime, the CKB platform conveys the acceptance information to the student who sent the invitation, ensuring a seamless interaction between both students involved.
Sofia and the sender of the invitation are promptly notified of the outcome. 
The invitation has been successfully accepted, and Sofia is now part of the team.



\paragraph*{Student rejects an invitation to join a team.}
Sofia, a dedicated student engaged in coding challenges on the CKB platform, is logged in and actively participating. 
Today, she receives a pop-up notification, signaling that she has received an invitation to join a team for a CKB.
After thoughtful consideration, Sofia decides to decline the invitation. 
With a click on the "Reject" button within the invitation pop-up, she communicates her decision to the CKB platform. 
Behind the scenes, the platform ensures this information is promptly relayed to the student who extended the invitation.
Both Sofia and the sender of the invitation receive immediate communication from the CKB platform, confirming the outcome of the declined invitation. 
The respectful rejection is acknowledged, allowing the students to proceed with their individual coding journeys.

\paragraph*{Student asks to join a team}
Anna is partecipating to a CKB tournament. She receives a notification by the system about an upcoming battle. 
Since the minimum team size is set to two for this particular battle, 
Anna can't take on the challenge by herself. 
However, she is a newbie to the platform and not really confident about creating a team. 
Therefore she decides to join a public team. Anna navigates 
to the battle page and clicks on the "Join Battle" button to subscribe to the challenge. The systems checks the minimum size to participate and asks wheter she wants to create a team or join one. 
Anna clicks the "Join Team" button. She is then prompted with the list of all avaibles public teams. 
She identifies a team called "The Code masters" with still one spot available 
and sends a request to join. The team leader (whoever created the team) is notified by the platform about this request and decides to accept the request. 
Anna is acknowledged, and added to the team.


\paragraph*{Student joins a battle without a team.}
Emma, an enthusiastic student with a passion for coding challenges, is actively participating in a coding tournament on the CKB platform. 
Excited about a specific battle, she navigates to the tournament page to explore the available challenges.
Spotting a captivating battle in the list, Emma decides to join without forming a team as she is confident on her own skills. 
She selects the desired battle and clicks the "Join Battle" button at the bottom of the list.
Behind the scenes, the CKB platform promptly verifies the registration deadline for the chosen battle. 
Upon completion of this process, the platform communicates the outcome of Emma's action. 
Whether met with success or needing adjustment, Emma receives immediate feedback, allowing her to seamlessly engage in the selected coding challenge.


\paragraph{Professor creates a tournament.}
Achille is a cyber security professor from university of Milan. He has an active profile on the CKB platform. 
During his last lesson, he annouced to his students that he will create monthly tournaments with weekly Battle for them to practise. 
He then proceeds to create the first tournament. He logs in the CKB platform from the "sign-in" page. 
Then, from the menu option in the home page, he selects "Create a new tournament". He is prompted with the page of the setup for the tournament itself. 
Achille enters a name for the tournament and the registration deadline. He then toggles the "advanced options" section. 
From here, he can decide wheter to include or not badges, and, in the first case, which badges are available in the tournament. 
Achille is prompted with a set of pre-existing badges (created from the platform or in previous tournaments). He can decide to 
create new badges in addition to the ones already present. When choosing to create a new badge, a pop-up window appears where he can select and combine 
variables and rules to obtain the new badge. Once he finishes setting up everything, he pushes the "Create tournament" button at the bottom of the page. 

\paragraph*{Professor closes a tournament.}
Logged into the CKB platform, Professor Lucia navigates to the "Tournaments" page. She locates the "Programming Prodigy Challenge" in the list and clicks
on it to access the tournament details. Satisfied with the students' participation and battle outcomes, she clicks the "Close Tournament" button.
A confirmation pop-up appears, and without hesitation, Professor Lucia confirms the action. The CKB platform processes the request, communicates 
the successful tournament closure, computes the final ranking based on accumulated scores, and notifies student about possible badges acquisition if necessary.
The platform promptly makes the final ranking available for participants. Professor Lucia, acknowledging the students' efforts, briefly reviews the rankings. 

\paragraph*{Professor creates a new CKB.}
Professor Roberto, logged into his CKB platform account, navigates to the tournament's details page named "Algorithmic Mastery Challenge." 
Excited to introduce a new coding challenge, he clicks on the "Create New Code Kata Battle" button. 
Professor Roberto is prompted to upload a code kata for the battle: he selects a well-prepared code kata that includes a comprehensive description and a software project with test cases and build automation scripts. 
He ensures that all necessary components are included before uploading.
Then he decides that groups should consist of a minimum of two students and a maximum of four students for this battle, so Professor Roberto sets these values accordingly in the "Group Size" field.
The educator proceeds to set a registration deadline and a final submission deadline, providing students with ample time to prepare and submit their solutions.
Curious about the advanced scoring options, Professor Roberto clicks on the "Additional Configurations for Scoring" button. 
The platform displays the scoring section, allowing him to set additional configurations if needed. 
Satisfied with the default scoring, he proceeds to click the "Create" button.

\paragraph*{Push action on GitHub}
Let’s consider a team named “CodeWarriors” consisting of two students, Alice and Bob. 
They are participating in a battle in one of the tournaments created by their professor, Achille. 
Alice and Bob are working on the code kata for the battle. They are in a brainstorming session, discussing the problem and proposing solutions. 
They write some initial code and test it locally on their machines. 
After some time, they are satisfied with their progress and decide to push their code to GitHub. 
Bob commits their code with a meaningful commit message and pushes it to their team’s repository on GitHub. 
As soon as the push action is performed, the CKB platform is triggered. 
It pulls the latest sources from the team’s repository, analyzes them, and runs the tests on the corresponding executables. 
The platform then calculates the team’s score based on the functional aspects, timeliness, and quality level of the sources. 
Alice and Bob receive a notification from the CKB platform about their updated score. 
They review the feedback, discuss the areas they need to improve, and continue working on the code kata. 
They are now in the process of refining their code and planning their next push action.

% We sure we want this to be possible???? 
\paragraph*{Professor grants permissions to another professor to add CKB to a tournament.}
Professor Marta, logged into her CKB platform account, navigates to the details page of the tournament named "Coding Challenge Extravaganza," which she initiated. 
Realizing the workload involved in creating battles, she decides to grant permissions to another educator, Professor Elena.
She clicks on the "Add Administrator" button, prompting the CKB platform to request the email or username of the educator to whom she wants to grant permissions. 
Professor Marta confidently inputs Professor Elena's email.
Professor Elena receives a notification from the CKB platform, informing her that she has been granted permissions to add battles to the tournament named "Coding Challenge Extravaganza."
Now, Professor Elena, can access the tournament details page and contribute by creating new Code Kata Battles. 

\paragraph*{Professor creates a new Badge.}
Professor Sofia, logged into her CKB platform account and currently on the details page of the "Coding Excellence Showcase" tournament, is inspired to introduce a special badge for outstanding performances. 
Excited about the idea, she clicks on the "Create New Badge" button located in the "Advanced Options" section.
The CKB platform prompts her to input the title of the badge. 
Professor Sofia, with a clear vision in mind, names the badge "Elite Coder."
Next, the platform allows Professor Sofia to define the rules for the badge. 
She navigates through a user-friendly wizard, creating rules that capture exceptional coding skills. 
Each rule represents a milestone that, when achieved, contributes to earning the "Elite Coder" badge.
Professor Sofia completes the rule-setting process, ensuring a fair and challenging criteria for students to meet. 
Satisfied with her creation, she clicks on the "Create" button.

\paragraph*{Professor manually evaluates teams in a CKB.}
Professor Marco, logged into his CKB platform account and currently on the details page of the "Algorithmic Challenge" CKB, a challenge that is ended and is in post evaluation phase.
The team already has a score, automatically computed by the platform, but Professor Marco wants to provide a more detailed evaluation. 
Eager to ensure a fair evaluation, he clicks on the "Manually Evaluate Teams" button.
The platform presents him with a list of teams that participated in the "Algorithmic Challenge." 
Professor Marco carefully selects a team from the list, keen on assessing their coding skills and approach to the given challenge.
Upon selecting a team, the CKB platform displays the team's submitted sources. 
Professor Marco thoroughly reviews the materials and, based on his expert judgment, assigns a comprehensive score to the team's performance.
Satisfied with the evaluation, Professor Marco clicks the "Save" button to record his assessment. 
This meticulous manual evaluation process repeats as Professor Marco proceeds to assess other teams involved in the "Algorithmic Challenge." 
Each score contributes to the overall ranking of the teams.

\paragraph*{User consults the leaderboard of a tournament.}
Sofia, an active user on the CodeKataBattle (CKB) platform, decides to check the current ranking of the "Data Structures Madness" tournament. 
She navigates to the "Tournaments" page and selects "Data Structure Madness" tournament.
Sofia proceeds opening the details page and effortlessly views the real-time leaderboard with a single click. 
The platform promptly presents the current standings, allowing Sofia to gauge the performance of each student subscribed to the tournament.

\paragraph*{User views the profile of another user to check their badges.}
Francesca, a student on the CodeKataBattle (CKB) platform, is intrigued to explore the profile of a student, Alex, to gain insights into his achievements. 
With a simple click on Alex's profile from the "Users" page, Francesca is presented with a comprehensive view. 
The platform seamlessly showcases the badges earned by Alex. 
To delve deeper, Francesca clicks on a specific badge and is presented with a detailed description of the badge and the criteria for earning it.

\paragraph*{User views the ranking of an ongoing CKB.}
Emma, an enthusiastic student on the CKB platform, is eager to witness the live dynamics of an ongoing coding tournament.
Logging into CKB, Emma navigates to a specific tournament, selecting a Code Kata Battle (CKB) of interest.
Within the chosen CKB, Emma clicks on "Ranking". In an instant, the CKB platform reveals the real-time rankings, showcasing the fluid movements of teams based on their pushes to the repository, automatically evaluated by the system.

%CodeKataBattle (CKB) is a new platform that helps students improve their software development skills
%by training with peers on code kata1 . Educators use the platform to challenge students by creating code
%kata battles in which teams of students can compete against each other, thus proving (and improving)
%their skills.

\subsection{State Diagrams}
\label{subsec:state_diagrams}%

Below are presented state diagrams relative to some of the scenarios described before

\begin{figure}[H]
    \begin{center}
        \includegraphics[width=0.9\linewidth]{Images/User-registration.jpg}
        \caption{User registration state diagram}
        \label{fig:state_diagram_1}%
    \end{center}
\end{figure}

\begin{figure}[H]
    \begin{center}
        \includegraphics[width=0.9\linewidth]{Images/join-team.jpg}
        \caption{Join team state diagram}
        \label{fig:state_diagram_4}%
    \end{center}
\end{figure}

\begin{figure}[H]
    \begin{center}
        \includegraphics[width=0.9\linewidth]{Images/student-invitation.jpg}
        \caption{Student invitation state diagram}
        \label{fig:state_diagram_5}%
    \end{center}
\end{figure}

\begin{figure}[H]
    \begin{center}
        \includegraphics[width=0.9\linewidth]{Images/Tournament_creation.jpg}
        \caption{Tournament creation state diagram}
        \label{fig:state_diagram_2}%
    \end{center}
\end{figure}

\begin{figure}[H]
    \begin{center}
        \includegraphics[width=0.9\linewidth]{Images/push-action.jpg}
        \caption{Push action state diagram}
        \label{fig:state_diagram_3}%
    \end{center}
\end{figure}


%Summary of major functions
\section{Product Functions}
\label{sec:product_functions}%
\textbf{CodeKataBattle}\\
\noindent\textit{CodeKataBattle} was born with the intention of helping students improve their software development skills by training with peers on code kata.
Educators use the platform to challenge students by creating code kata battles in which teams of students can compete against each other, thus proving (and improving) their skills.\\
\subsection*{Tournaments and Battles}
CodeKataBattle (CKB) platform offers a dynamic and interactive service for creating tournaments and battles that plays a pivotal role in fostering collaborative coding endeavors. 
Educators leverage this functionality to design engaging coding competitions between students.
CKB platform handles tournaments as a collection of battles, each with a specific set of parameters and rules. Each tournament has its ranking, which consist of a leaderboard of single students.
Battles hold their own ranking, which consist of a leaderboard of teams. 
The platform provides a user-friendly interface that allows educators to create tournaments and battles, set up scoring systems, and manage the overall tournament dynamics.
Students, in response, actively participate in competing in battles, laying the groundwork for collaborative problem-solving experiences.

\subsection*{Team Formation}
The team formation feature within the CodeKataBattle (CKB) platform stands as a pivotal element, streamlining the collaborative dynamics of problem-solving among students. 
This integral aspect of the platform caters to an array of user scenarios, presenting students with versatile options to either autonomously join battles or take a proactive role by extending and receiving invitations. 
This flexibility empowers students to shape their teams based on preferences, expertise, or collaborative strategies.
The platform ensures a user-friendly team creation process, allowing students to seamlessly navigate the formation landscape. 
The invitation system is a precision tool, enabling users to send tailored invitations to specific individuals. 
It adheres to the battle-specific constraints, considering the predefined minimum and maximum number of students per group. 
This precision minimizes ambiguity in team size, ensuring a cohesive and efficient team-building process.
Moreover, the acceptance mechanism is seamlessly integrated, facilitating swift and hassle-free team assembly. 
The platform's intuitive design considers user experience at its core, ensuring that the team formation process is not only functional but also enhances the overall collaborative learning experience.

\subsection*{Automated Scoring for Code Evaluation}
Automated scoring is a core feature, evaluating various facets of student submissions. 
The platform is capable of performing automated analysis on the student' submissions.
This analysis covers functional aspects, including test case pass rates, timeliness of submissions, and the quality of source code.
The quality of the source code is determined by multiple factors, including security, reliability, and maintainability.
Educators can leverage automated tools for static code analysis to ensure a comprehensive evaluation but are also able to manually evaluate submissions if needed.

\subsection*{Notifications}
Automated notifications are dispatched to students when educators close or create a tournament. 
These notifications include final tournament ranks, providing closure to the competition and acknowledging the achievements of participating students.
Moreover, the platform sends notifications to students when they are invited to join a team or when their invitation is accepted or rejected.

\subsection*{Gamification Badges}
To inject a gamified element, educators define badges to recognize and reward student achievements. 
Badges, with titles like "Top Performer" or "Code Guru," serve as visual representations of accomplishments, enhancing student profiles and fostering a sense of achievement.
Educators can include badges in tournaments by selecting from a list of pre-existing badges or creating new ones.
When creating new badges, educators must define rules and variables that determine when a badge is awarded to a student.
Rules and variables can either be selected from a list of pre-existing ones or created from scratch.
In order to create a new variable, educators can make use of metrics exposed by the platform that offer information that could be relevant for scoring. 
These metrics can be combined to arithmetic operations to create new variables.
New rules can be defined by performing operation on variables (avreage, max, min, sum, etc.), resulting in complex requirements the students must meet to earn the badge.



\section{User characteristics}
\label{sec:user_characteristics}%
The CKB platform caters to various user roles, each with specific responsibilities and access levels:
\begin{itemize}
    \item \textbf{Guest:}  Users who are not registered on the CKB platform but have the potential to become either students or educators. 
    Guests have limited access and functionality until they complete the registration process. 
    Once registered, they can assume the roles of either Student or Educator based on their role in the institution they belong to.
    \item \textbf{Educator:} Educators who have successfully registered within the CKB system. Identified by a unique identifier, registered educators, depending on permissions granted, may assume the role of Tournament Administrator. 
    They possess the ability to create CKBs, evaluate student submissions, and engage in other system functionalities.
    \item \textbf{Student:} Users, both registered and unregistered, participating in code kata battles facilitated by the CKB platform. 
    Students can form teams, submit solutions, and engage in the competition. 
    Their scores, rankings, and achievements contribute to their overall performance, visible in the context of tournaments.
    \item \textbf{Tournament Administrator:} Educators with privileges to perform actions within a specific Tournament. Tournament Administrators, typically the creators of a Tournament, can delegate administrative powers to other educators. 
    These administrators have the authority to initiate new CKBs, evaluate submissions, and oversee tournament-related activities.
\end{itemize}

\section{Assumptions, Dependencies, and Constraints}
\label{subsec:Assumptions, Dependencies, and Constraints}%

\subsection{Assumptions}

\newcounter{ac}
\setcounter{ac}{0}
\newcommand{\ca}{\stepcounter{ac}\theac}

\newcommand{\myrow}[1]{
    A\ca & #1 \\
    \hline
}

\begin{center}
    \begin{longtable}{ |l|p{0.9\linewidth}| }
        \hline
        \textbf{ID} & \textbf{Assumptions}                                                                   \\
        \hline
        A\ca        & Educators using CKB platform have the necessary technical knowledge and skills to create and manage code kata battles. 
        This includes the ability to create programming exercises, write test cases, and set up build automation scripts.                                   \\
        \hline
        A\ca        & All users of the CKB platform, both educators and students, are assumed to have access to a stable internet connection. This is
         necessary fir accessing the platform, downloading code kata, submitting code, and receiving notification.                                  \\
        \hline
        A\ca        & Educators are expected to have the ability to evaluate the work done by students and assign scores if manual evaluation is required. \\
        \hline
        A\ca        & Educators managing a torunament will not lose access to their institutional email during the whole duration of the tournament itself.                                                             \\
        \hline
        A\ca        & Educators and students have a good understanding of GitHub and GitHub actions.                                  \\
        \hline
        A\ca        & Educators are capable of creating and closing tournaments, without leaving them open undefinetly.                      \\
        \hline
        \caption{Assumptions.}
        \label{tab:assumption_tab}%
    \end{longtable}
\end{center}

\subsection{Constraints}

\newcounter{cc}
\setcounter{cc}{0}
\newcommand{\ccc}{\stepcounter{cc}\theac}

\newcommand{\ccrow}[1]{
    C\ccc & #1 \\
    \hline
}

\begin{center}
    \begin{longtable}{ |l|p{0.9\linewidth}| }
        \hline
        \textbf{ID} & \textbf{Assumptions}.                                   \\
        \hline
        C\ccc        & Educators and students are linked to one and only one institutional email. There can't be multiple accounts owned by the same person..                                  \\
        \hline
        C\ccc        & Students can participate in a battle with one and only one group or alone. It is not possible to have a student signed 
        for a battle with more than one group or one group and individually. \\
        \hline
        C\ccc        & Students will have to write their code in one of the language supported by the CKB platform.                                                             \\
        \hline
        C\ccc        & Educators can't close a tournament while there is still a battle unravelling in that specific torunament.                                  \\
        \hline
        C\ccc        & For each battle, the number of students per group is constrained by the minimum and maximum number set by 
        the educator during the creation of the battle. This means that a group cannot have fewer members 
        than the minimum limit or more members than the maximum limit.                      \\
        \hline
        C\ccc        & The submission of code by students is constrained by the final submission deadlines set by the educator. Any submissions made after the final submission deadline will not be considered for evaluation.                                                \\
        \hline
        \caption{Constraints.}
        \label{tab:constraints_tab}%
    \end{longtable}
\end{center}

\subsection{Dependencies}

\newcounter{dc}
\setcounter{dc}{0}
\newcommand{\cd}{\stepcounter{dc}\theac}

\newcommand{\dcrow}[1]{
    C\cd & #1 \\
    \hline
}

\begin{center}
    \begin{longtable}{ |l|p{0.9\linewidth}| }
        \hline
        \textbf{ID} & \textbf{Assumptions}.                                   \\
        \hline
        D\cd        & Affiliated universities have an automated system to provide data about an authenticated student/educator to the CKB platform.                                  \\
        \hline
        D\cd        &  The operation of the CodeKataBattle (CKB) platform heavily depends on its integration with GitHub. 
        GitHub is used for code submission, version control, and tracking commits made by students. \\
        \hline
        D\cd        & The notification of students about new tournaments, battles, ranks, and badges depends on the notification system of the CKB platform. 
        The platform must be able to send timely and accurate notifications to all subscribed students.                                                             \\
        \hline
        \caption{Dependencies.}
        \label{tab:dependencies_tab}%
    \end{longtable}
\end{center}