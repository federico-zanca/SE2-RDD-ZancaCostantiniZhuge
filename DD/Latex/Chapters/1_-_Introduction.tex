\section{Purpose}
\label{sec:purpose}
This document provides a comprehensive overview of the CodeKataBattle (\verb|CKB|) platform, elucidating the strategies to meet the project requirements delineated in the Requirements Analysis and Specification Document (RASD). 
Tailored for developers and testers, this document serves as a guide, offering a functional delineation of the system's components, their interactions, interfaces, and the planned implementation strategies.
For developers, the document acts as a guide, providing a closer look at the details of CKB's architecture and design choices.
Testers will find valuable information on the expected functionalities and the interaction patterns within the system.
Furthermore, this document serves as a bridge between the abstraction of the concepts in the RASD and the concrete implementation detailed here. 
It addresses each requirement outlined in the RASD, explaining how the various system components, described herein, collectively fulfill these requirements. 
This document aims to enhance comprehension of how the \verb|CKB| platform meets the specified project objectives by detailing its internal workings and external interfaces.

\section{Scope}
\label{sec:scope}
The Design Document for the CodeKataBattle (CKB) platform provides a comprehensive overview of the system's architecture, 
design, and functionality. It details the interfaces of various components such as the UserManager and TournamentManager, 
which are integral to the operation of the platform.\newline
The document also elaborates on the structural design and pattern choices, 
including the use of a firewall and a reverse proxy for security, and the deployment of 
two databases (one SQL and one NoSQL) for efficient data management.\newline
The interface design is meticulously outlined to ensure a user-friendly experience, 
facilitating seamless interaction between users and the platform.\newline
The document also includes a section on requirement traceability, ensuring that every 
requirement defined in the RASD is accounted for in the design and implementation phases. 
This helps in maintaining the coherence and completeness of the project.\newline
The implementation, integration, and testing sections provide a roadmap for the development process, 
outlining the strategies for integrating the various components and the approaches for testing the system's functionality and performance.\newline
As described in the RASD, CKB is a platform designed to enhance software development skills through competitive coding challenges. 
It allows users to participate in code kata battles, where they compete against each other in tournaments by solving programming exercises. 
The platform collects the solutions, allows educators to create battles and evaluate the solutions, and updates the scores and 
rankings of the teams based on the evaluation.\newline
Incorporating gamification elements like badges, CKB motivates users and enhances their learning experience. 
Educators can create badges with custom rules and assign them to users based on their performance. 
The platform also supports the creation of tournaments, where multiple battles are grouped together, 
and users compete for the highest cumulative score.\newline
In essence, CKB is a comprehensive platform for competitive coding challenges, providing a dynamic, engaging, 
and educational environment for users to improve their coding skills. The Design Document serves as a blueprint 
for realizing this vision, detailing the system's design and implementation strategies to ensure the successful execution of the project.\newline

\section{Definition, Acronyms, Abbreviations}
\label{sec:definition_acronyms_abbreviations}%
\begin{table}[H]
    \begin{center}
        \begin{tabular}{ |l|l| }
            \hline
            \textbf{Acronyms} & \textbf{Definition}            \\
            \hline
            CKB              & Code Kata Battle    \\
            \hline
            RASD               & Requirements Analysis and Specification Document       \\
            \hline
            DD                & Design Document                \\
            \hline
            WPX               & World Phenomena X              \\
            \hline
            SPX               & Shared Phenomena X             \\
            \hline
            GX                & Goal Number X                  \\
            \hline
            DAX               & Domain Assumptions X           \\
            \hline
            UCX               & Use Case X                     \\
            \hline
            DBMS               & Database Management System        \\
            \hline
            UI                & User Interface               \\
            \hline
            API               & Application Programming Interface \\
            \hline
            OS               & Operating System               \\
            %\hline
            %REST              & Representational State Transfer               \\
            \hline
            HTTP             & Hypertext Transfer Protocol               \\
            \hline
            UX              & User Experience               \\
            \hline
        \end{tabular}
        \caption{Acronyms used in the document.}
        \label{tab:acronyms}%
    \end{center}
\end{table}

\section{Revision History}
\label{sec:revision_history}
This is the first version of this document.

\section{Reference Documents}
\label{sec:reference_documents}%
\begin{itemize}
    \item \href{https://polimi365-my.sharepoint.com/:b:/g/personal/10710351_polimi_it/EZXUPFfeFKdBkf5M8W-EBYgB2JrrVLr23BYJ4MXQ7kzUkA?e=o0wvyw}{The specification document Assignment RDD AY 2023-2024.pdf}
    \item {RASD document}
\end{itemize}

\section{Document Structure}
\label{sec:document_structure}

The structure of this Design Document is organized into seven distinct sections, each contributing to a comprehensive understanding of the CodeKataBattle (CKB) platform.

The opening section (Introduction) serves as a preamble, introducing the purposes and objectives of the document. 
It also includes a compilation of abbreviations and definitions used to enhance the readability of the document.

The second section (Architectural Design) delves into the chosen architectural design for CKB. 
It describes the identified system components, their interrelations, communication interfaces, behavioral aspects within the system, and the utilization of architectural styles.

The third section (User Interface Design) focuses on user interfaces, presenting and elucidating the mockups designed to enhance the user experience on the CKB platform.

Section four (Requirements Traceability) systematically demonstrates how the CKB system aligns with the defined requirements. 
It begins by establishing a mapping between identified components and the functional requirements outlined in the third section of the RASD. 
Subsequently, a detailed account of the fulfillment of performance requirements and system attributes is provided.

The fifth section (Implementation, Integration, and Test Plan) outlines the plan for implementing, integrating, and testing various subcomponents of the CKB system. 
It details the sequence in which subcomponents will be implemented and integrated, setting the groundwork for the practical realization of the platform.

Section six (Effort Spent) provides transparency regarding the effort invested by each group member in the creation of this document. 

The concluding section serves as a repository of bibliography references and additional resources utilized in the creation of this Design Document. 
