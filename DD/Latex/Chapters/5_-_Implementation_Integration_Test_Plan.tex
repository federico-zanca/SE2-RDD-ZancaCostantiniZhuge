\section{Implementation}
The implementation process begins with setting up the databases. MongoDB, a NoSQL database, is chosen for its flexibility and scalability, 
making it ideal for storing diverse data types such as users, educators, tournament results, badges, notifications, and more. 
This database will be designed with a schema-less structure, allowing for easy modifications as the data requirements evolve over time.\newline
Concurrently, MySQL, a SQL database, is used for its strength in handling structured data and complex queries. 
It will store data related to tournaments, teams, and battles. Given the nature of these data types, they are likely to involve numerous queries, 
especially during an ongoing tournament. The structured nature of SQL databases allows for efficient querying and retrieval of data, 
ensuring smooth operation during high-load scenarios.\newline
Once the databases are set up, the backend development begins. 
This involves creating core functionalities such as user manager, battle manager, tournament manager, badge manager, and the notification system. 
Each of these components is crucial for the functioning of CKB and will involve interactions with the databases. 
For instance, user manager will handle user registration, login, and profile management, interacting primarily with the MongoDB database. 
Battle manager, on the other hand, will involve interaction with the MySQL database. The order in which each components needs to be developed is not a strict one. 
Here, we present a suggestion on the order.\newline
We first suggest creating the user manager, since this will allows us to immediately test subscription and login, as well as interaction with MongoDB. 
Then, it would be auspicious to develop the Tournament manager. This entity relies on the existence of an Educator to create and manage the Tournament. 
Therefore, this is purely a logical step in the development of the back end. 
Once the Tournament manager is properly functioning, we can focus on the production of the Battle manager and Team manager, two entities strictly related to the 
Tournament Manager. Then, all the other aspects can be creating without a specific order, since the core of the logic of the backend is already up.\newline
Last steps for the backend development will be the integration of the CKB platform with all the external tools required to achieve the desired implementation. 
This includes integrating the notification system with the mail system, the communication with GitHub, and the implementation with Google Spreadsheets and Universities 
Authentication.\newline
Following the completion of the backend, the focus shifts to frontend development. 
This includes designing and implementing the user interface for various operations such as viewing and joining battles, 
viewing scores, user registration, login, and profile management. 
Additionally, real-time updates using WebSocket will be set up for live scoring and notifications of new battles or tournaments. 
This feature is crucial for enhancing the user experience by providing immediate feedback and updates.\newline
To ensure the security of the application, a firewall is set up to protect the servers from potential cyber attacks and unauthorized access. 
The firewall is configured to allow WebSocket traffic. In addition, a reverse proxy is set up for load balancing, SSL termination, and simplified network routing. 
These measures enhance the security and performance of the application.\newline
Throughout the implementation process, careful management is required to handle data synchronization and maintain data integrity across both databases. 
A reliable mechanism is implemented to keep the databases updated and consistent with each other. 
This comprehensive implementation plan provides a robust and adaptable data management solution for CKB, ensuring its efficiency and reliability.

\section{Integration}
The integration phase is a critical step in the development process of CodeKataBattle (CKB), where the individual components of the system, 
developed during the implementation phase, are combined and tested as a whole.\newline
The integration process begins with the combination of the frontend and backend components of CKB. 
This involves connecting the backend with the user interface elements in the frontend. 
The WebSocket connection between the client and the server, set up during the implementation phase, is also integrated into the system to enable real-time updates.\newline
Next, the databases are integrated with the backend. 
This involves setting up the database connections and ensuring that the backend can correctly query the databases and handle the returned data. 
The MongoDB database, which stores non-relational data such as users, educators, tournament results, badges, notifications, and more, 
is integrated with the user management, badge management, and notification system components of the backend. 
The MySQL database, which stores relational data related to tournaments, teams, and battles, is integrated with the battle management 
and tournament management components of the backend.\newline
The firewall and reverse proxy, set up during the implementation phase for security and performance enhancement, 
are also integrated into the system. The firewall is configured to allow WebSocket traffic, 
which is essential for the real-time updates in CKB. The reverse proxy is set up for load balancing, SSL termination, and simplified network routing.\newline
The final step in the integration process is the integration of the various components of the backend. 
This includes integrating the user management, battle management, tournament management, badge management, 
and notification system components with each other. This ensures that these components can work together seamlessly to provide the core functionalities of CKB.

\section{Testing}
\subsection{Unit Testing}
Unit testing is a critical process in software development, where individual components of a program are tested in isolation to ensure they function correctly. 
This method focuses on the smallest units of code, like functions or methods, to identify issues early, enhancing overall software quality. 
This type of test will be performed in the early stage of the development. In particular, 
during the implementation of each manager, unit testing will be required in order to ensure that each component as a standalone works properly.

\subsection{Integration Testing}
Integration testing is a level of software testing where individual units are combined and tested as a group. 
The goal is to identify any problems or bugs that arise when different components are combined and interact with each other.\newline
The integration testing can follow the order of the implementation of the back end logic, and be conducted in parallel with it 
(this means implementation, integration and testing steps for the back end are done simultaneously). 
This will ensure that everything works properly and as intended. If an error arises at this point of the testing, it will be easier to spot and correct it, 
instead of waiting for testing after the end of the implementation and integration.

\subsection{System Testing}
System testing validates the complete and fully integrated software product. 
The purpose is to evaluate the end-to-end system specifications. 
System testing will require to verify every input in the application to check for desired outputs, 
users’ experience and integration as a whole including external services

\subsection{Performance Testing}
Performance testing ensures software applications perform properly under their expected workload. 
It focuses on evaluating the performance and scalability of a system. 
The goal is to identify bottlenecks, measure system performance under various loads and conditions, 
and ensure the system can handle the expected number of users or transactions. 
In particular, we want to make sure in this phase that even under stress, they system can still provide real time updates to the users, and that no data is lost.
